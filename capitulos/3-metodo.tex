%%%% CAPÍTULO 5 - Metodologia

\chapter{Metodologia}\label{cap:Metodologia}

\section{Questões de pesquisa}

\section{Método}
    \item Realizar uma revisão da literatura abrangente, explorando estudos relevantes em \gls{sds} e métricas de código em \gls{es}. OBS: Ir para a metodologia!
    \item Selecionar exemplos específicos de código trabalhados nos repositórios escolhidos, aplicando a heurística desenvolvida e justificando a seleção com base em métricas de código. OBS: Passo dos materiais e métodos.
    \item Avaliar a utilidade educacional dos exemplos de código selecionados, usando-os em um contexto de ensino. OBS: Passo dos materiais e métodos
    \item Selecionar métricas de código específicas e relevantes apropriadas para avaliar a evolução de código em \gls{sds}, considerando as melhores práticas da área. OBS: Parece um passo.
    \item Selecionar um ou mais repositórios \textit{open-source} pertinentes à área de \gls{sds} que servirão como fonte de exemplos de código para análise.

\section{Abordagem proposta}

\section{Ferramentas}\label{sec:ferramentas}
O \gls{ck} \cite{aniche-ck} é uma ferramenta poderosa que nos permite analisar a qualidade do código-fonte em projetos Java por meio da análise de métricas estáticas de código. Essas métricas fornecem informações valiosas sobre características como complexidade, acoplamento e coesão de classes. Ao incorporar o \gls{ck} em nossa pesquisa, estamos criando formas de classificar classes ou métodos com relação a métricas estáticas de qualidade de código. A ferramenta em questão apresenta mais de 35 métricas.

O RefactoringMiner \cite{Tsantalis:ICSE:2018:RefactoringMiner} é uma ferramenta fundamental no contexto da \gls{es}, especializada em identificar e analisar refatorações de código-fonte. Essa ferramenta desempenha um papel crucial ao fornecer uma compreensão aprofundada das mudanças realizadas em um código ao longo do tempo. Com a sua capacidade de reconhecer padrões de refatoração, o RefactoringMiner permite que os desenvolvedores analisem como o código foi modificado, de modo a melhorar a qualidade, manutenibilidade e eficiência do software.

A biblioteca PyDriller \cite{PyDrillerSpadini2018} é um recurso essencial para análise de repositórios de código-fonte escritos em Python. Sua funcionalidade abrangente permite aos pesquisadores e desenvolvedores explorar e compreender a evolução do código em projetos Python de uma maneira eficiente. O PyDriller oferece recursos para extrair informações cruciais, como histórico, autoria e mensagens de \textit{commits}, além de detalhes específicos do estado do código em um determinado momento, contribuindo assim para a pesquisa e prática em \gls{es} no contexto de sistemas complexos e interconectados.

OBS: Pedir ajuda com a repetição da palavra "código".

\section{Repositórios}\label{sec:repositorios}
Falar do Kafka e do Zookeeper, explicar cada um deles e justificar a escolha desses SDs.

\section{Métodos}\label{sec:metodo}

\section{Resultados esperados}


\section{Cronograma}