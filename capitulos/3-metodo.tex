%%%% CAPÍTULO 5 - Metodologia

\chapter{Metodologia}\label{cap:Metodologia}

\section{Questões de pesquisa}

\section{Método}

\section{Abordagem proposta}

\section{Ferramentas}\label{sec:ferramentas}
O \gls{ck} é uma ferramenta poderosa que nos permite analisar a qualidade do código-fonte em projetos Java por meio da análise de métricas estáticas de código. Essas métricas fornecem informações valiosas sobre características como complexidade, acoplamento e coesão de classes. Ao incorporar o \gls{ck} em nossa pesquisa, estamos criando formas de classificar classes ou métodos com relação a métricas estáticas de qualidade de código. A ferramenta em questão apresenta mais de 35 métricas, sendo que no estudo foram consideradas as seguintes:
\begin{itemize}
    \item \gls{cbo}: Consiste na definição do grau de acoplamento (dependências) que uma determinada classe apresenta. Quanto maior for o valor do \gls{cbo}, maior é o grau de acoplamento da classe, o que indica maior interdependência entre classes. Isso pode tornar o código mais complexo e menos flexível, uma vez que alterações na classe afetaria o comportamento de inúmeras outras classes. Dessa forma, para um determinado código, dada a sua evolução, quando observada uma queda nesse valor, tem-se então um bom indício de melhorias relevantes na qualidade daquele código.
    \item \gls{cbom}: Métrica similar a anterior, todavia considera também a dependência de classes uma referência a um objeto do tipo, ou seja, ao adicionar uma simples chamada de um método da classe, essa métrica é incrementada. Intuitivamente, o \gls{cbom} foi utilizado nos estágios iniciais do estudo foi retirado por poluir os valores, visto que o escopo de refatorações nas métricas foi menos restrito a refatorações da classe ou do método.
    \item \gls{rfc}: Refere-se ao número de invocações únicas de um método de uma determinada classe, isto é, a métrica conta o número de invocações estáticas. 
    OBS: Analisar e explicar o impacto dos valores dessa métrica.
    \item \gls{wmc}: OBS: Analisar e explicar o impacto dos valores dessa métrica de acordo com \cite{MetricsSuite}
\end{itemize}

RefactoringMiner.
PyDriller.

OBS: Pedir ajuda com a repetição da palavra "código".

\section{Repositórios}\label{sec:repositorios}
Falar do Kafka e do Zookeeper, explicar cada um deles e justificar a escolha desses SDs.

\section{Métodos}\label{sec:metodo}

\section{Resultados esperados}


\section{Cronograma}