%%%% CAPÍTULO 5 - Metodologia

\chapter{Metodologia}\label{cap:Metodologia}

\section{Questões de pesquisa}

\section{Método}

\section{Abordagem proposta}

\section{Ferramentas}\label{sec:ferramentas}
O \gls{ck} é uma ferramenta poderosa que nos permite analisar a qualidade do código-fonte em projetos Java por meio da análise de métricas estáticas de código. Essas métricas fornecem informações valiosas sobre características como complexidade, acoplamento e coesão de classes. Ao incorporar o \gls{ck} em nossa pesquisa, estamos criando formas de classificar classes ou métodos com relação a métricas estáticas de qualidade de código. A ferramenta em questão apresenta mais de 35 métricas.

O RefactoringMiner é uma ferramenta fundamental no contexto da \gls{es}, especializada em identificar e analisar refatorações de código-fonte. Essa ferramenta desempenha um papel crucial ao fornecer uma compreensão aprofundada das mudanças realizadas em um código ao longo do tempo. Com a sua capacidade de reconhecer padrões de refatoração, o RefactoringMiner permite que os desenvolvedores analisem como o código foi modificado, de modo a melhorar a qualidade, manutenibilidade e eficiência do software, contribuindo assim para a pesquisa e prática em \gls{es} no contexto de sistemas complexos e interconectados.

PyDriller.

OBS: Pedir ajuda com a repetição da palavra "código".

\section{Repositórios}\label{sec:repositorios}
Falar do Kafka e do Zookeeper, explicar cada um deles e justificar a escolha desses SDs.

\section{Métodos}\label{sec:metodo}

\section{Resultados esperados}


\section{Cronograma}