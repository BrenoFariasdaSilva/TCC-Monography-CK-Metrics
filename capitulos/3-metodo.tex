%%%% CAPÍTULO 5 - Metodologia

\chapter{Metodologia}\label{cap:Metodologia}
Neste capítulo, descreveremos as ferramentas e métodos que empregaremos para conduzir nossa pesquisa. Abordaremos as escolhas fundamentais que orientam nossa abordagem, incluindo as técnicas de coleta e análise de dados, as ferramentas computacionais e quaisquer modelos ou teorias relevantes. A metodologia consiste na estrutura que sustenta nosso estudo e nos permite atingir nossos objetivos de pesquisa.

\section{Questões de pesquisa}


\section{Método}



\section{Abordagem proposta}




\section{Ferramentas}\label{sec:ferramentas}
As ferramentas usadas 
\begin{itemize}
    \item CK - Code Metrics: O CK é uma ferramenta poderosa que nos permite analisar a qualidade do código-fonte em projetos Java por meio da análise de métricas de código. Essas métricas fornecem informações valiosas sobre características como complexidade, acoplamento e coesão de classes. Ao incorporar o CK em nossa metodologia, estamos criando formas de classificar classes ou métodos com relação a métricas estáticas de qualidade de código. A ferramenta em questão apresenta mais de 35 métricas disponíveis, todavía nosso estudo considerou relevante o uso das seguintes métricas:
    \begin{itemize}
        \item \gls{cbo}: Consiste na definição do grau de acoplamento dependências que uma determinada classe apresenta. Quanto maior for o valor do \gls{cbo}, maior é o grau de acoplamento da classe, o que indica uma maior interdependência entre classes. Isso pode tornar o código mais complexo e menos flexível, uma vez que alterações na classe afetaria o comportamento de inúmeras outra classes. Dessa forma, para um determinado código, dada a sua evolução, quando observada uma queda nesse valor, tem-se então um bom indício de melhorias relevantes na qualidade daquele código.
        \item \gls{cbom}: Métrica similar a anterior, todavía considera também a dependência de classes como sendo uma referência a um objeto do tipo, ou seja, ao adicionar uma simples chamada de um método da classe, essa métrica é incrementada. Intuitivamente, o \gls{cbom} foi utilizado nos estágios iniciais do estudo, todavía foi retirado, pois acabou por poluir os valores, visto que o escopo de refatorações nas métricas foi menos restrito a refatorações da classe ou do método.
        \item \gls{rfc}: Refere ao número de invocações únicas de um método de uma determinada classe, onde essa métrica conta o número de invcações estáticas. OBS: Analisar o impacto dos valores dessa métrica.
        \item \gls{wmc}: OBS: Analisar o impacto dos valores dessa métrica.
    \end{itemize}
    \item RefactoringMiner.
    \item PyDriller.
\end{itemize}

OBS: Pedir ajuda com a repetição da palavra "código".

\section{Métodos}\label{sec:metodo}



\section{Resultados esperados}



\section{Cronograma}