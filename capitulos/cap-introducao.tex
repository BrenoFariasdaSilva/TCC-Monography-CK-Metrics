%%%% CAPÍTULO 1 - INTRODUÇÃO
%%
%% Deve apresentar uma visão global da pesquisa, incluindo: breve histórico, importância e justificativa da escolha do tema,
%% delimitações do assunto, formulação de hipóteses e objetivos da pesquisa e estrutura do trabalho.

%% Título e rótulo de capítulo (rótulos não devem conter caracteres especiais, acentuados ou cedilha)
\chapter{Introdução}\label{cap:introducao}

Neste capítulo adentraremos ao cerne da pesquisa, fornecendo uma visão geral do contexto e objetivos. Além disso, será apresentada uma justificativa do objeto de pesquisa trabalho, delineamento o escopo e a estrutura do trabalho a ser apresentado, bem como uma breve explicação da estrutura na qual este trabalho é composto. 

\section{Considerações iniciais}\label{sec:consideracoesIniciais}

\section{Objetivos}\label{sec:objetivos}

\subsection{Objetivo geral}\label{subsec:objetivoGeral}

Prover uma heurística a qual servirá para encontrar bons exemplos trabalhados de evolução de código por meio da análise de métricas CK. Dessa forma, uma vez conquistada uma heurística definida, pretende-se escolher um bom exemplo de modo a servir como material a ser utilizado em sala de aula, o qual irá proporcionar uma análise de como, no mercado de trabalho, softwares focados na área de Sistemas Distribuídos evoluem ao longo do tempo.

\subsection{Objetivos específicos}\label{subsec:objetivosEspecificos}

\begin{itemize}
    \item Escolher repositório(s) open-source apropriado(s) relacionado(s) à area de Sistemas Distribuídos.
    \item Seleção de ferramentas para geração de métricas do(s) repositório(s) selecionado(s).
    \item Seleção das métricas relevantes para a heurística a ser desenvolvida.
    \item Análise dos dados para validação da heurística.
    \item Refinamento da heurística.
    \item Seleção de possíveis bons exemplos e estudo do que é possível retirar de boas refatoraçôes.
\end{itemize}

\section{Justificativa}\label{sec:justificativa}

\section{Estrutura do trabalho}\label{sec:estruturaTrabalho}
