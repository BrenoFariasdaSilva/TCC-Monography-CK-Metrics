%%%% CAPÍTULO 5 - Conclusões

\chapter{Conclusões}
\label{cap:conclusoes}

Neste trabalho, o qual representa uma proposta de trabalho para o TCC2, foram desenvolvidas ferramentas e uma heurística em andamento para analisar a evolução de projetos de \gls{sds} por meio da avaliação de métricas de código. As métricas selecionadas inicialmente, como \gls{cbo}, \gls{rfc}, e \gls{wmc}, foram escolhidas para proporcionar uma visão abrangente da qualidade do código, cobrindo diferentes aspectos. A métrica \gls{cbom} foi removida durante o desenvolvimento do estudo devido à percepção de que poderia introduzir poluição nos dados.

Para facilitar a análise da evolução do código, cinco ferramentas específicas foram desenvolvidas: \textit{code\_metrics.py}, \textit{metrics\_changes.py}, \textit{track\_files.py}, \textit{metrics\_evolution\_refactors.py} e \textit{repository\_refactors.py}. Todas essas ferramentas foram implementadas em Python, utilizando o \textit{PyDriller} e o \textit{RefactoringMiner}, e estão disponíveis no Github\footnote{https://github.com/BrenoFariasdaSilva/Scientific-Research}.

A heurística em desenvolvimento visa oferecer uma abordagem sistemática e contextualizada para a análise da evolução do código em \gls{sds}, reconhecendo as complexidades e desafios inerentes a essa tarefa. No entanto, a conclusão definitiva dessa heurística está prevista para o TCC2.

Embora os resultados prévios não sejam tão animadores quanto o esperado, eles fornecem \textit{insights} valiosos para a continuidade da pesquisa. A falta de padrão nos \textit{commits} e o elevado número de refatorações por \textit{commit}, muitas vezes não diretamente alinhadas à mensagem do \textit{commit}, representam desafios significativos a serem abordados.

Este trabalho estabeleceu uma base sólida com um referencial teórico abrangente, um cronograma bem definido e uma metodologia detalhada. As fases da pesquisa, desde a revisão da literatura até a avaliação da utilidade educacional dos exemplos escolhidos, foram claramente delineadas no cronograma, proporcionando uma abordagem sistemática e eficiente na realização dos objetivos propostos, além de expor completo conhecimento das limitações e desafios que este trabalho está imerso.

Em resumo, a proposta deste estudo busca preencher uma lacuna na literatura sobre a criação de exemplos trabalhados em \gls{es}, utilizando \gls{sds} como objeto de estudo. Ao contribuir para a formação de estudantes e profissionais em \gls{es}, espera-se que a compreensão mais profunda da evolução do código em \gls{sds} e os exemplos trabalhados pedagogicamente valiosos impactem positivamente o ensino e a prática profissional nesta área.