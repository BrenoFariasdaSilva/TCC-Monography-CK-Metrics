\chapter{Conclusões}
\label{chapter:conclusions}

O presente trabalho de pesquisa buscou investigar a evolução do código em \gls{sds} por meio da análise de métricas de código, com o intuito de criar exemplos trabalhados aplicáveis ao ensino de \gls{es}. A pesquisa foi orientada por quatro questões principais, delineando uma abordagem exploratória e qualitativa para compreender a relação entre métricas de código e melhorias em \gls{sds}, bem como para desenvolver exemplos trabalhados pedagogicamente eficazes para a \gls{es}.

Na formulação das questões de pesquisa, destacamos a importância de identificar métricas relevantes para analisar a evolução em \gls{sds}, observar padrões e tendências associados a melhorias no código, avaliar o impacto das métricas escolhidas em melhorias de desempenho, segurança, entre outros, e, finalmente, transformar essas melhorias em exemplos trabalhados para o ensino de \gls{es}.

A abordagem proposta fundamentou-se na compreensão da complexidade inerente aos \gls{sds} na era digital, justificando a necessidade de estudar a evolução do código para formar profissionais qualificados. A heurística desenvolvida para a seleção de códigos representativos de melhorias mostrou-se vital para delimitar o escopo de código a ser analisado, de modo a tentar criar exemplos trabalhados que ilustram, de forma prática, a dinâmica da evolução do código em \gls{sds}.

A escolha criteriosa das ferramentas, como CK, RefactoringMiner e PyDriller, permitiu uma análise abrangente das métricas de código e das mudanças ao longo do tempo nos repositórios selecionados. Destacamos a importância do Apache Kafka e ZooKeeper como repositórios relevantes, dada sua significativa presença em \gls{sds}, proporcionando um ambiente propício para a análise da evolução do código em \gls{sds}.

A metodologia adotada pretende possibilitar uma compreensão holística da relação entre métricas específicas e melhorias no código, indo além de uma abordagem quantitativa para incorporar elementos qualitativos na análise. A seleção de exemplos de código trabalhados baseados na heurística desenvolvida visa contribuir diretamente para o ensino de \gls{es}, fornecendo material prático e aplicável aos estudantes.

Os resultados prévios e o cronograma delinearam as fases da pesquisa, desde a revisão da literatura até a avaliação da utilidade educacional dos exemplos escolhidos. O cronograma ofereceu uma estrutura clara para as atividades planejadas ao longo dos meses, assegurando uma abordagem sistemática e eficiente na realização dos objetivos da pesquisa. Atualmente, nossos resultados destacam desafios, alguns dos quais eram esperados desde o início do projeto, relacionados à falta de padrão nos \textit{commits} e ao grande número de refatorações por \textit{commit}, muitas vezes não diretamente alinhadas à mensagem do \textit{commit}.

Em síntese, este trabalho visa preencher uma lacuna na literatura sobre a criação de exemplos trabalhados em \gls{es}, utilizando os \gls{sds} como objeto de estudo, contribuindo para a formação de estudantes e profissionais em \gls{es}. Ao proporcionar uma compreensão mais profunda da evolução do código em \gls{sds} e oferecer exemplos trabalhados pedagogicamente valiosos, espera-se impactar positivamente o ensino e a prática profissional na área de \gls{es}.
