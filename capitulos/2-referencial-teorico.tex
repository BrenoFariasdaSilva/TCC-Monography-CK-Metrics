%%%% CAPÍTULO 2 - REVISÃO DA LITERATURA (OU REVISÃO BIBLIOGRÁFICA, ESTADO DA ARTE, ESTADO DO CONHECIMENTO)
%%
%% O autor deve registrar seu conhecimento sobre a literatura básica do assunto, discutindo e comentando a informação já publicada.
%% A revisão deve ser apresentada, preferencialmente, em ordem cronológica e por blocos de assunto, procurando mostrar a evolução do tema.
%% Título e rótulo de capítulo (rótulos não devem conter caracteres especiais, acentuados ou cedilha)

\chapter{Referencial teórico}
\label{cap:referencialTeorico}

Este capítulo oferece uma base conceitual para compreender os principais elementos de \gls{sds} e seus desafios. São explorados conceitos acerca de exemplos trabalhados, além de expor alguns estudos sobre sua utilização. Tais estudos englobam não apenas a educação em geral, mas também na área de \gls{es}. Dessa forma, este capítulo visa fornecer um contexto abrangente que sustentará a análise e discussão dos resultados obtidos ao longo deste trabalho.

\section{Sistemas Distribuídos}
\label{section:sistemasDistribuidos}

% Conceito de um Sistema Distribuído.
Um \gls{sd} é aquele no qual os componentes localizados em computadores interligados em rede se comunicam e coordenam suas ações apenas passando mensagens. A motivação para o desenvolvimento de \gls{sds} é impulsionada pela necessidade de melhorar a eficiência na utilização de recursos computacionais, possibilitando que a carga de trabalho seja dividia entre vários servidores, oferecendo suporte a aplicações multitarefa e fornecendo maior confiabilidade por meio da redundância de dados e serviços. A facilidade de acesso remoto a recursos também é uma motivação significativa, permitindo que usuários acessem informações e serviços independentemente da localização física dos recursos \cite{TanenbaumDistributedSystemsThirdEdition}.

% Contexto sobre Sistemas Distribuídos em sala de aula.
% \subsection{Educação em sistemas distribuídos}
% Eventos específicos sobre educação em computação sobre sistemas distribuídos:
% - EduPar: NSF/TCPP Workshop on Parallel and Distributed Computing Education (https://tcpp.cs.gsu.edu/curriculum/?q=edupar23)
% - EduHiPC: Workshop on Education for High Performance Computing (https://tcpp.cs.gsu.edu/curriculum/?q=eduHiPC23)

% Estudo sobre Sistemas Distribuídos no âmbito educacional.
Um estudo abrangente foi conduzido em 2019 e apresentado no artigo ``\textit{Have We Reached Consensus? An Analysis of Distributed Systems Syllabi}'' \cite{HaveWeReachedConsensus}. Nele, a grade das 51 disciplinas mais famosas sobre \gls{sds} de programas de Ciência da Computação foram analisadas, conforme as melhores universidades pelo \textit{Times Higher Education}, visando identificar os principais tópicos abordados. A análise desses dados revelou que conceitos fundamentais, como Processos e Threads, Replicação, Chamadas de Sistema, Controle de Concorrência, Tolerância a Falhas, Sincronização, Comunicação, entre outros, constituem os elementos-chave frequentemente lecionados em disciplinas de \gls{sds}. Além disso, a pesquisa abordou a questão das referências utilizadas na elaboração do conteúdo dos cursos, identificando que obras como ``\textit{Distributed Systems}'' na edição de 2017, de Van Steen e Tanenbaum, e ``D\textit{istributed Systems: Concepts and Design}'' na edição de 2011, de Coulouris, são fontes comuns amplamente utilizadas para explicar os princípios dos \gls{sds} \cite{HaveWeReachedConsensus}.

% Desafios em Sistemas Distribuídos.
A implementação de \gls{sds} enfrenta diversos desafios, sendo um deles a heterogeneidade, que se manifesta em diferentes níveis, como heterogeneidade de hardware, rede, linguagem de programação e implementações realizadas por diferentes desenvolvedores. Essa diversidade pode exigir o uso de \textit{middlewares} para fornecer uma camada de abstração e facilitar a integração. A ausência de um relógio global em \gls{sds} introduz desafios na coordenação de eventos e sincronização entre diferentes partes do sistema. A tolerância a falhas é abordada por meio de técnicas como \textit{checksums} para detecção de erros, mascaramento de falhas usando retransmissão de dados, além de contar com o uso de operações de \textit{rollback} para manter a consistência, mesmo diante de falhas \cite{DistributedSystemsCoulouris}.

% Desafios em Sistemas Distribuídos - Continuação.
Para atingir escalabilidade, \gls{sds} empregam técnicas como replicação de dados, uso de cache distribuída, distribuição de carga entre múltiplos servidores e utilização de \textit{web proxies}, entre outros. No entanto, a escalabilidade também traz consigo desafios, como a necessidade de manter a consistência entre as réplicas de dados. Adicionalmente, o Teorema \gls{cap} estabelece que é impossível garantir fortemente, de forma simultânea, características de Consistência, Disponibilidade e Tolerância a partição em um \gls{sd}. Assim, durante o desenvolvimento de um \gls{sd}, é preciso fazer um balanço entre diferentes objetivos \cite{TwentyYearsLaterEricBrewer2012}.

% Desafios em Sistemas Distribuídos - Continuação.
A troca de mensagens \gls{ipc} é uma questão pertinente em \gls{sds}, podendo ocorrer de forma síncrona ou assíncrona, além de ser classificada como bloqueante ou não bloqueante, dependendo do comportamento desejado. Do mesmo modo, a questão da coordenação e do acordo, na eleição de nó líder, por exemplo, também são fundamentais, com técnicas como eleição, protocolos de consenso e algoritmos de ordenação de eventos desempenham papéis cruciais na operação confiável de \gls{sds}. Em específico, a ordenação de eventos é tarefa particularmente complexa pela ausência de um relógio global, a qual na maioria das vezes é baseada em relações \textit{happened-before} \cite{DistributedSystemsCoulouris}.

% Desafios em Sistemas Distribuídos - Continuação.
% Resumindo os desafios e vantagens de utilizar Sistemas Distribuídos como objeto de estudo na Engenharia de Software.
Em resumo, o desenvolvimento de \gls{sds} é desafiador, o que evidencia a importância do estudo desses sistemas. A complexidade desses desafios destaca a importância da \gls{es} na busca por soluções melhores. Ao analisar a evolução do código em \gls{sds} por meio de estudos de casos, a \gls{es} pode extrair \textit{insights} valiosos para a aplicação prática de conceitos teóricos e estratégias de resolução de problemas. A utilização de \gls{sds} como cenário de estudo oferece uma base sólida para construir exemplos trabalhados, enriquecendo o ensino em \gls{es}. A interseção entre os desafios de \gls{sds} e os princípios da \gls{es} cria uma oportunidade única para desenvolver materiais de ensino que promovam uma compreensão aprofundada, preparando os estudantes para enfrentar as complexidades do desenvolvimento de software.

\section{Exemplos trabalhados}

% Lacuna na literatura.
Os estudos existentes, como ``\textit{Learning from Examples - Instructional Principles from the Worked Examples Research}'' \cite{Robert.Atkinson-etal:2000}, ``\textit{Using Real Worked Examples to Aid Software Engineering Teaching}' \cite{Simone.Tonhao-etal:2021}, ``Uma Plataforma Gamificada de Desafios Baseados Em \textit{Worked Examples} Extraídos de Projetos de Software Livre Para o Ensino de Engenharia de Software'' \cite{Simone.Tonhao-etal:2022} e ``\textit{Students' Perception of Example-Based Learning in Software Modeling Education}'' \cite{Tiago.Bonetti-etal:2023} comprovam que o uso de exemplos trabalhados, quando aplicados corretamente, potencializam o engajamento e retenção do conteúdo exposto. Entretanto, a maioria desses trabalhos se concentra principalmente na \gls{es}, o que pode ser problemático quando consideramos contextos que envolvem sistemas de informação ou \gls{sds} mais simples, como cliente-servidor. 

% Definição de exemplo trabalhado.
Conforme exposto no artigo ``\textit{Learning from Examples - Instructional Principles from the Worked Examples Research}'' \cite{Robert.Atkinson-etal:2000}, um exemplo trabalhado é definido como um trabalho cognitivo e experimental com o intuito de fornecer uma solução ideal, todavia próxima ao praticável, para um problema específico, no qual uma pessoa com escasso ou nenhum conhecimento acerca do tema possa examinar e aprender com a solução proposta. Sendo assim, o desenvolvimento e estudo de exemplos trabalhados enriquecem a qualidade do que é lecionado em sala de aula, uma vez que a solução ideal pode representar muito bem o estado da arte para um determinado tópico, visto que dissemina conceitos e padrões do problema apresentado \cite{Robert.Atkinson-etal:2000}.

% Conceitos sobre o uso de exemplos trabalhados em sala de aula.
No mesmo contexto, os principais elementos na elaboração de um exemplo trabalhado de alta qualidade estão intrinsecamente vinculados aos conceitos de \textit{``inter-example feature''} e \textit{``intra-example feature''}. A concepção do termo \textit{``inter-example feature''} destaca a importância de fornecer uma diversidade de exemplos que ilustrem estratégias múltiplas para problemas similares, porém de tipos diferentes, enquanto o termo \textit{``intra-example feature''} trata de características intrínsecas à estrutura interna de cada exemplo \cite{Robert.Atkinson-etal:2000}. 

% Definição de exemplo trabalhado em educação em computação.
O artigo intitulado ``\textit{Worked Examples in Computer Science}'' \cite{Skudder-LuxtonReilly:2014} destaca a escassez de pesquisas sobre exemplos trabalhados na área da Ciência da Computação. Além disso, o trabalho expõe que as tarefas de programação requerem um alto teor cognitivo, logo isso pode ser um indício para haver poucos trabalhos estudando exemplos trabalhados nesta área. Porém, o uso de exemplos trabalhados seria benéfico, visto que diminui a carga cognitiva para compressão do conteúdo \cite{Robert.Atkinson-etal:2000}. Adicionalmente, o autor investiga as ramificações de exemplos trabalhados na área da Computação, frequentemente manifestados na apresentação passo-a-passo da execução de um código específico, prática recorrente em disciplinas como análise de algoritmos. Não suficiente, o estudo também explora o emprego do conceito de ``\textit{problem solution pair}'', no qual um problema é apresentado, estimulando o aluno a tentar resolvê-lo, seguido pela exposição da solução correspondente. Por fim, de maneira menos convencional, são utilizadas técnicas de desenvolvimento de software enquanto se leciona um conteúdo, o que contribui para esclarecer o raciocínio empregado na elaboração da solução para o problema.

% Estratégias de uso de exemplos trabalhados em sala de aula.
A exposição dos alunos a um determinado conteúdo, como já vista, pode ser feita pelo uso de exemplos trabalhados \cite{Robert.Atkinson-etal:2000}. Todavia, o uso de \textit{``Faded worked examples''} é uma estratégia de ensino que apresenta exemplos de problemas resolvidos aos alunos de forma gradual. Inicialmente, fornece-se um exemplo completamente resolvido para garantir a compreensão total da solução. Em seguida, ao longo de problemas subsequentes, partes do exemplo resolvido são progressivamente removidas. Os alunos são desafiados a preencher as lacunas, completando as partes ausentes da solução. Isto implica em abster o discente de lidar com situações passivas, tornando-o proficiente em resgatar informações no hipocampo do cérebro \cite{Skudder-LuxtonReilly:2014}.

% Impacto do uso de exemplos trabalhados.
O impacto dos exemplos trabalhados na aprendizagem varia conforme o nível de experiência do aprendiz. Para iniciantes, esses exemplos proveem benefícios ao frisar a atenção nos elementos cruciais do problema, facilitando a criação de esquemas de resolução pertinentes e otimizando recursos cognitivos em comparação com a abordagem direta na resolução de problemas. Em contraste, para indivíduos com certa experiência, os exemplos trabalhados podem tornar-se supérfluos, resultando em uma carga cognitiva extrínseca \cite{Skudder-LuxtonReilly:2014}.

\section{Exemplos trabalhados na Engenharia de Software}

% Exemplos trabalhados para engenharia de software.
No estado atual da arte, poucos trabalhos são encontrados sobre o uso de exemplos trabalhados em \gls{es}, muito menos em \gls{sds}. Porém, o artigo intitulado ``\textit{Using Real Worked Examples to Aid Software Engineering Teaching}'' \cite{Simone.Tonhao-etal:2021} busca compreender o uso de exemplos trabalhados na \gls{es}. Para tal, é realizado um \textit{survey} que evidenciou haver pouco uso de exemplos trabalhados reais em sala de aula, devido às dificuldades encontradas na criação dos mesmos. Um questionário também foi feito para os discentes, onde foi apontado que o uso de exemplos trabalhados reais resultou em uma melhoria na percepção e motivação dos alunos.

No mesmo contexto, em um estudo de caso em sala de aula, foram utilizados exemplos trabalhados como material para ensinar a aplicação real de conceitos e para criação de atividades. Os resultados desse estudo de caso mostraram que mais de 80\% dos estudantes ficaram mais motivados, sentiram que melhoraram as suas habilidades para o mercado de trabalho e, não suficiente, mais de 90\% dos discentes concordaram que os exemplos trabalhados foram relevantes para a qualidade do ensino \cite{Simone.Tonhao-etal:2021}.

Um exemplo trabalhado, conforme definido por \citeonline[p. 1]{Simone.Tonhao-etal:2021}, é um instrumento que compreende a apresentação de um problema, as etapas envolvidas em sua resolução e o desfecho final. Para ilustrar, consideremos um exemplo utilizado no estudo, onde foi feita a explicação do conceito da anomalia \textit{large class}, que se refere a uma classe que assume mais responsabilidades do que seria apropriado. Esse problema pode ser identificado por meio da métrica \gls{loc} ou pela análise de coesão da classe. Em seguida, o contexto do problema em um projeto específico é delineado, no qual uma classe chamada \textit{PackWrite} é examinada. Esta classe, responsável por gerar arquivos de pacote para um conjunto específico de objetos do repositório, incorporava uma classe estática, \textit{ObjectToPack}, encarregada de empacotar um objeto.

Para solucionar essa questão, é necessário extrair a classe \textit{ObjectToPack} para uma nova classe, tornando-a independente da classe \textit{PackWriter} e transferindo todas as suas funcionalidades. Como resultado desse processo, a classe \textit{PackWrite} passou a aderir ao princípio de responsabilidade única, no qual uma classe deve ser dedicada a resolver um problema específico, garantindo, assim, sua coesão integral. Dessa forma, o estudo mostrou, de forma eficiente, o que é um exemplo trabalhado, além do fato deles contribuírem grandiosamente para a qualidade do ensino \cite{Simone.Tonhao-etal:2021}. No entanto, a criação de exemplos trabalhados não é trivial e apresenta diversos desafios.

% Desafios em usar exemplos trabalhados em sala de aula.
A criação ou busca de exemplos trabalhados pode ser (e geralmente é) bem complexa. Uma das conclusões obtidas pela análise do \textit{survey} realizado pelo estudo anterior aponta que os professores em geral têm dificuldade em encontrar bons exemplos, de acordo com: a complexidade desejada, adequação à disciplina lecionada, tópico procurado, entre outras. As principais dificuldades apontadas são a adequação à disciplina e a complexidade, visto que muitos exemplos são: (i) muito grandes e complexos, ou (ii) muito triviais, sem qualquer conexão com a realidade \cite{Simone.Tonhao-etal:2021}. Além disso, outra dificuldade relevante apontada está relacionada ao fato de muitos exemplos não serem representativos do mundo real, por estarem desatualizados. Já é conhecido por outro estudo que um fator relevante para que alunos tenham uma boa ideia do que é exigido no mercado de trabalho é o uso de projetos atualizados \cite{OSSProjects-TheProfessors'Perspective}.

% Estudo sobre vantagens do uso de exemplos trabalhados em sala de aula.
Por meio de uma pesquisa exploratória feita no trabalho ``\textit{Students’ Perception of Example-Based Learning in Software Modeling Education}'' \cite{Tiago.Bonetti-etal:2023}, foi realizado um experimento com professores que nunca haviam utilizado \gls{abe}. Professores selecionados implementarem esta abordagem em uma aula sobre modelagem de diagramas de classe. A implementação da \gls{abe} foi feita instruindo os professores sobre como a abordagem proposta funciona. Em seguida, os professores deviam construir o plano de aula, utilizando o portal disponibilizado, para então executarem o plano de aula. A referida aula foi sucedida por um questionário que visava avaliar as percepções dos aprendizes em relação ao conteúdo apresentado. 

Ainda sobre o mesmo estudo, a análise qualitativa dos resultados teve como objetivo observar os aspectos benéficos e desafiadores desse experimento. Os resultados evidenciaram que a estratégia de \gls{abe} não apenas contribuiu significativamente para o grau de compreensão e retenção dos conceitos abordados, mas também facilitou uma maior conexão entre a teoria e a prática. Adicionalmente, o trabalho destacou, como perspectivas futuras, a necessidade de investigar o impacto dessas estratégias em outras áreas da \gls{es}.

\section{Trabalhos relacionados}
\label{section:related-work}

% Ferramenta para análise de Sistemas Distribuídos.
O trabalho ``\textit{DistFax - A Toolkit for Measuring Interprocess Communications and Quality of Distributed Systems}''\cite{DistFax} tem um cunho de análise de métricas em \gls{sds}, porém não com o intuito de criar exemplos trabalhados, mas sim de fornecer uma ferramenta que ajude a analisar determinados aspectos de \gls{sds}. Dessa forma, até o momento, não foram encontradas pesquisas sobre a utilização da evolução de código-fonte de \gls{sds} por meio de métricas estáticas de código para a criação de exemplos trabalhados na disciplina de \gls{es}.

No mesmo contexto, o artigo propõe a ferramenta \textit{DistFax} para a análise de métricas de \gls{ipc}, que demonstraram correlação com métricas de qualidade de software. As métricas utilizadas na ferramenta são dinâmicas, as quais requerem o uso de relações \textit{happened-before} que, como já explicado no começo desse capítulo, são um desafio por não haver um relógio global.

% Ferramenta para análise de Sistemas Distribuídos - Continuação.
% Métricas de acoplamento.
As métricas de acoplamento utilizadas no \textit{DistFax} são:
\begin{itemize}
    \item \textbf{\gls{rmc}}: Contagem de mensagens enviadas de um processo para outro.
    \item \textbf{\gls{rcc}}: Razão entre o número de métodos em uma classe dependentes de métodos em outra classe.
    \item \textbf{\gls{ccc}}: Representa o acoplamento agregado de uma classe em relação a classes em processos remotos.
    \item \textbf{\gls{ipr}}: Mede o acoplamento entre processos mediante métodos comuns.
    \item \textbf{\gls{ccl}}: Avalia a carga de comunicação de uma classe individual com outras classes em todos os processos remotos.
\end{itemize}

% Ferramenta para análise de Sistemas Distribuídos - Continuação.
% Conceito de Complexidade Ciclomática.
A complexidade ciclomática, que representa a quantidade de caminhos distintos no código, pode ser interpretada como o número mínimo de casos de teste necessário para cobrir todos os caminhos do programa sem repetição, servindo para garantir uma cobertura adequada durante os testes. Portanto, quanto menor a complexidade ciclomática, melhor.

% Ferramenta para análise de Sistemas Distribuídos - Continuação.
% Modelos gerados e métricas disponíveis.
No mesmo contexto, dois modelos de \gls{ia} foram desenvolvidos, para predizer a condição de qualidade de um \gls{sds}, dado como entrada as métricas \gls{ipc} coletadas, onde o modelo supervisionado utilizava a técnica de \textit{Bagging} e o modelo não supervisionado utilizava o algoritmo \textit{K-Means}. No \textit{DistFax}, foram utilizadas seis métricas de \gls{ipc} (cinco de acoplamento e uma de coesão), correlacionadas com cinco métricas de qualidade de software, sendo elas, tempo de execução, complexidade ciclomática, contagem de caminhos de fluxo de informações, comprimento do caminho do fluxo de informações e superfície de ataque. Observaram-se correlações positivas entre: (i) \gls{ccc} com o tempo de execução, (ii) \gls{rmc} e \gls{ccc} com a complexidade ciclomática, e (iii) \gls{ccc} com a área de ataque (do inglês, \textit{attack-surface}). Houve também uma correlação negativa entre \gls{ccl} e \gls{plc} com a área de ataque \cite{DistFax}.

% Ferramenta para análise de Sistemas Distribuídos - Continuação.
% Métrica de coesão.
A única métrica de coesão utilizada no \textit{DistFax} foi \gls{plc}, que mede as conexões internas em um processo individual.

% Ferramenta para análise de Sistemas Distribuídos - Continuação.
% Correlação entre métricas e seus significados.
A correlação positiva entre \gls{ccc} e o tempo de execução sugere que um elevado nível de acoplamento está correlacionado a um desempenho inferior. Portanto, nas métricas \gls{ck} adotadas em nossa pesquisa, reconhecemos que \gls{cbo}, o qual será explicado na seção \ref{sec:metricasDeSoftware}, desempenha um papel importante. Além disso, a correlação positiva entre \gls{ccc} e a área de ataque, em termos de segurança de software, indica que códigos com alto acoplamento são mais vulneráveis a potenciais pontos de ataque. Isso significa que um invasor pode explorar esses pontos para obter acesso não autorizado ou realizar ações prejudiciais. Esse fato reforça a importância da métrica \gls{cbo} como um indicador significativo da qualidade do código, especialmente no contexto da segurança do software.

O artigo ``RefactorScore: Evaluating Refactor Prone Code''\cite{KevinJesse:2023} desenvolve a ferramenta intitulada \textit{RefactorScore}\footnote{https://huggingface.co/kevinjesse/RefactorBERT/tree/main}, a qual tem o intuito de prover uma métrica de avaliação automática de código que reconhece, de acordo com os \textit{tokens} de um arquivo, áreas propensas a refatoração. Com base nessa identificação, a ferramenta gera uma pontuação correspondente, refletindo o potencial de melhoria nessas áreas específicas. A ferramenta \textit{RefactorScore} está presente no modelo \textit{RefactorBert}, o qual aprendeu com a mudança de códigos entre refatorações, mostrando-se eficiente em detectar erros de qualidade de código. Esse tipo de erro está, por exemplo, associado a código fora do padrão de \textit{design} de código adotado. 

O modelo \textit{RefactorBert} foi desenvolvido com base em repositórios de linguagem C/C++\cite{C_CppProgrammingLanguage}, contendo mais de um milhão de \textit{commits} ao todo. Contudo, os repositórios examinados nesta monografia não estão presentes no dataset\footnote{https://huggingface.co/datasets/kevinjesse/ManyRefactors4C} utilizado pela ferramenta \textit{RefactorScore}, o que torna impraticável seu uso para a coleta de dados relacionados à refatoração nos contextos abordados. Adicionalmente, a falta de documentação sobre a utilização da ferramenta \textit{RefactorScore}, juntamente com a limitada presença de apenas três \textit{commits} no repositório da ferramenta, reforça a inviabilidade de seu emprego. Para adaptar a ferramenta ao contexto de repositórios Java\cite{JavaProgrammingLanguage}, seria essencial criar um novo \textit{dataset} e treinar o modelo para reconhecer padrões de refatoração em código-fonte do tipo almejado.

\section{Considerações finais}
Este capítulo explorou o referencial teórico necessário para compreender os \gls{sds} (\gls{sds}) e seus desafios, fornecendo uma base conceitual sólida para a análise e discussão dos resultados desta pesquisa. Os principais Conceitos foram destacados, como a definição de \gls{sds}, os desafios inerentes à sua implementação, e a interseção entre esses desafios e os princípios da \gls{es}.

Ao analisar os desafios específicos enfrentados pelos \gls{sds}, destacamos a importância da heterogeneidade, a ausência de um relógio global, a necessidade de tolerância a falhas e escalabilidade. A discussão sobre o Teorema \gls{cap} ressaltou os \textit{trade-offs} cruciais na busca por consistência, disponibilidade e tolerância a partições em um \gls{sd}.

Explorou-se também o conceito de exemplos trabalhados e sua relevância no contexto da educação em \gls{es}. Os exemplos trabalhados oferecem uma abordagem eficaz para ensinar conceitos complexos, proporcionando uma ponte entre a teoria e a prática. No entanto, a criação e seleção de exemplos trabalhados apresentam desafios, especialmente no contexto de \gls{sds}.

Além disso, trabalhos relacionados que abordam ferramentas e métricas para análise de qualidade em \gls{sds} foram discutidos, destacando o \textit{DistFax} e suas métricas de acoplamento, coesão e correlações significativas com desempenho e segurança do software.

No contexto educacional, evidenciou-se a escassez de pesquisas sobre exemplos trabalhados em \gls{sds}, ressaltando a importância de abordagens pedagógicas que integrem de maneira eficaz os desafios específicos desses sistemas.

No próximo capítulo, é apresentada a metodologia adotada para a realização deste estudo, detalhando os procedimentos e técnicas utilizados na coleta e análise dos dados.
