%%%% CAPÍTULO 2 - REVISÃO DA LITERATURA (OU REVISÃO BIBLIOGRÁFICA, ESTADO DA ARTE, ESTADO DO CONHECIMENTO)
%%
%% O autor deve registrar seu conhecimento sobre a literatura básica do assunto, discutindo e comentando a informação já publicada.
%% A revisão deve ser apresentada, preferencialmente, em ordem cronológica e por blocos de assunto, procurando mostrar a evolução do tema.
%% Título e rótulo de capítulo (rótulos não devem conter caracteres especiais, acentuados ou cedilha)
\chapter{Referencial teórico}
\label{cap:referencialTeorico}


\section{Fundamentação teórica}
\label{section:background}
Falar sobre \gls{sds} usando o \cite{DistributedSystemsCoulouris} como base

\subsection{Exemplos trabalhados}

% Definição de exemplo trabalhado
A definição de exemplo trabalhado (do inglês, \textit{worked example}) encontra-se claramente difundida na literatura, passível de ser compreendido como um trabalho cognitivo e experimental com o intuito de fornecer uma solução ideal, todavia próxima ao praticável, para um problema específico, no qual uma pessoa com escasso ou nenhum conhecimento acerca do tema possa examinar e aprender com a solução proposta. Sendo assim, o desenvolvimento e estudo de exemplos trabalhados enriquecem a qualidade do que é lecionado em sala de aula, uma vez que a solução ideal pode representar muito bem o estado da arte para um determinado tópico, visto que dissemina conceitos e padrões do problema apresentado \cite{Robert.Atkinson-etal:2000}.

Conforme exposto no artigo \textit{``Learning from Examples - Instructional Principles from the Worked Examples Research''} \cite{Robert.Atkinson-etal:2000}, os elementos essenciais na elaboração de um exemplo trabalhado de alta qualidade estão intrinsecamente vinculados aos conceitos de \textit{``inter-example feature''} e \textit{``intra-example feature''}. A concepção do termo \textit{``inter-example feature''} destaca a importância de fornecer uma diversidade de exemplos que ilustrem estratégias múltiplas para problemas similares, porém de tipos diferentes, enquanto o termo \textit{``intra-example feature''} insinua características intrínsecas à estrutura interna de cada exemplo. 

% Definição de exemplo trabalhado em educação em computação
O artigo intitulado ``Worked Examples in Computer Science'' \cite{Skudder-LuxtonReilly:2014} destaca a escassez de pesquisas sobre exemplos trabalhados na área da Ciência da Computação. Isso é evidenciado pelo fato de que apenas uma minoria das referências citadas no artigo aborda efetivamente exemplos trabalhados relacionados à Ciência da Computação, uma vez que as tarefas de programação requerem um alto teor cognitivo. Além disso, na presente pesquisa, o autor investiga as ramificações de exemplos trabalhados na área da Computação, frequentemente manifestados na apresentação passo a passo da execução de um código específico, prática recorrente em disciplinas como análise de algoritmos. Adicionalmente, o estudo explora o emprego do conceito de ``problem solution pair'', no qual um problema é apresentado, estimulando o aluno a tentar resolvê-lo, seguido pela exposição da solução correspondente. Por fim, de maneira menos convencional, são utilizadas técnicas de desenvolvimento de software enquanto se leciona um conteúdo, o que contribui para esclarecer o raciocínio empregado na elaboração da solução para o problema.

A exposição dos alunos a um determinado conteúdo, como já vista, pode ser feita pelo uso de exemplos trabalhados, o qual diminui a carga cognitiva necessária para compreensão de um determinado tópico \cite{Robert.Atkinson-etal:2000}, todavia o uso de \textit{Faded worked examples} é uma estratégia de ensino que apresenta exemplos de problemas resolvidos aos alunos de forma gradual. Inicialmente, fornece-se um exemplo completamente resolvido para garantir a compreensão total da solução. Em seguida, ao longo de problemas subsequentes, partes do exemplo resolvido são progressivamente removidas ou "desvanecidas". Os alunos são desafiados a preencher as lacunas, completando as partes ausentes da solução, implicando em abster o discente de lidar com situações passivas e torna-se proficiente em resgatar informações no hipocampo do cérebro \cite{Skudder-LuxtonReilly:2014}.

O impacto dos exemplos trabalhados na aprendizagem varia conforme o nível de experiência do aprendiz. Para iniciantes, esses exemplos proveem benefícios ao frisar a atenção nos elementos cruciais do problema, facilitando a criação de esquemas de resolução pertinentes e otimizando recursos cognitivos em comparação com a abordagem direta na resolução de problemas. Em contraste, para indivíduos com certa experiência, os exemplos trabalhados podem tornar-se supérfluos, resultando em uma carga cognitiva extrínseca \cite{Skudder-LuxtonReilly:2014}.

% Exemplos trabalhos para engenharia de software
No contexto contemporâneo, deparamo-nos com um desafio significativo associado à evolução incessante do conhecimento em domínios como Medicina, Física e, notavelmente, a Computação. Nesse contexto, disciplinas como (\gls{sd}) e (\gls{es}) apresentam desafios substanciais tanto para discentes quanto para docentes. Considerando a contribuição positiva do uso de exemplos trabalhados para a qualidade do ensino, torna-se imperativo analisar a perspectiva do educador. Pesquisas indicam que os pedagogos enfrentam obstáculos ao selecionar e desenvolver exemplos trabalhados, além de evidenciar a falta de pesquisa sobre exemplos trabalhados na \gls{es}. Acrescentando a isso, é importante ressaltar que apenas uma minoria dos professores faz uso de exemplos reais, apesar de quase a totalidade incorporar exemplos em suas práticas pedagógicas, conforme evidenciado por um \textit{survey} realizado no artigo intitulado ``Using Real Worked Examples to Aid Software Engineering Teaching'' \cite{Simone.Tonhao-etal:2021}.

Ademais, ao abordar a exposição dos alunos a projetos \textit{open-source}, um dos principais desafios reside na seleção de projetos apropriados. Idealmente, um projeto ativo com múltiplos contribuidores externos seria uma escolha favorável. No entanto, o maior desafio reside em engajar o estudante para realizar uma contribuição significativa, uma vez que projetos reais demandam um nível técnico considerável \cite{OSSProjects-TheProfessors'Perspective}. Não suficiente, a forma em que o exemplo trabalhado é exposto também tem relevância, dessa forma a apresentação do exemplo trabalho por meio de uma gamificação também pode apresentar um impacto positivo no nível de engajamento dos discentes \cite{Simone.Tonhao-etal:2022}.

Por meio de uma pesquisa exploratória, professores que nunca haviam utilizado \gls{abe}
implementaram seu uso em uma aula sobre modelagem de diagramas de classe. A referida aula foi sucedida por um questionário que visava avaliar as percepções dos aprendizes em relação ao conteúdo apresentado. A análise qualitativa dos resultados teve como objetivo observar os aspectos benéficos e desafiadores desse experimento. Os resultados evidenciaram que a estratégia de \gls{abe} não apenas contribuiu significativamente para o grau de compreensão e retenção dos conceitos abordados, mas também facilitou uma maior conexão entre a teoria e a prática. Adicionalmente, o presente trabalho destacou, como perspectivas futuras, a necessidade de investigar o impacto dessas estratégias em outras áreas da \gls{es} \cite{Tiago.Bonetti-etal:2023}.

\subsection{Métricas de software}
O artigo \textit{"A Metrics Suite for Object Oriented Design"} \cite{MetricsSuite} apresenta um conjunto de métricas para avaliação do design de código orientado a objetos, introduzindo métricas cruciais para compreender a complexidade e qualidade do código-fonte. Dentre elas, destacam-se o \gls{cbo}, \gls{rfc} e \gls{wmc}. Essas métricas fornecem compreensões valiosas sobre a estrutura e evolução de sistemas orientados a objetos, sendo essenciais para avaliações de qualidade de software. Além dessas métricas, o artigo define outras, como o \gls{dit}, \gls{noc} e \gls{lcom}, ampliando o conjunto de ferramentas disponíveis para a análise do design orientado a objetos. Neste contexto, exploraremos cada uma dessas métricas, destacando sua importância e contribuição para a avaliação de qualidade de software.

\begin{itemize}
    \item \textbf{\gls{cbo}}: Reflete o grau de acoplamento, ou seja, as dependências que uma classe específica possui, indicando a quantidade de classes diretamente associadas a essa classe, já que utiliza métodos dessas classes. Um valor elevado sugere maior complexidade e menor flexibilidade, pois mudanças na classe podem impactar várias outras, não suficiente, demasiado acoplamento diminui o grau de modularidade de uma classe. Observar uma redução no \gls{cbo} ao longo da evolução do código é um indicativo positivo de melhorias na qualidade do código.
    \item \textbf{\gls{cbom}}: Similar ao \gls{cbo}, entretanto, o \gls{cbom} considera a dependência de classes como uma referência a um objeto do mesmo tipo. Isso significa que, ao adicionar uma simples chamada de método da classe, a métrica é incrementada. No entanto, durante os estágios iniciais do estudo, o \gls{cbom} foi removido por poluir os valores, pois o escopo de refatorações nas métricas era menos restrito a refatorações da classe ou do método. É importante observar que essa métrica não faz parte do conjunto original do \gls{ck}, sendo introduzida devido a outra interpretação da métrica \gls{cbo}.
    \item \textbf{\gls{rfc}}: Refere-se ao número de invocações únicas de um método de uma determinada classe, isto é, a métrica conta o número de invocações estáticas, avaliando o nível de comunicação entre classes. Um alto valor de métodos que podem ser invocados em função de dar uma resposta a um método, maior será a dificuldade em testar e depurar o código em questão.
    \item \textbf{\gls{wmc}}: Simboliza a soma dos valores de complexidade dos métodos de uma classe específica. Um valor elevado indica que a classe pode ser complexa, com múltiplos métodos, implicando em um custo significativo para o desenvolvimento e manutenção. Além disso, a existência de vários métodos associados a uma classe sugere que ela pode ser menos genérica, sem mencionar o possível impacto nos filhos dessa classe.
\end{itemize}
\subsection{Educação em sistemas distribuídos}

\section{Trabalhos relacionados}
\label{section:related-work}

DistFax: A Toolkit for Measuring Interprocess Communications and Quality of Distributed Systems.
\cite{DistFax}
