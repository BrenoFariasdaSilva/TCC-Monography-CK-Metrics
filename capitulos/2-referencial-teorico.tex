%%%% CAPÍTULO 2 - REVISÃO DA LITERATURA (OU REVISÃO BIBLIOGRÁFICA, ESTADO DA ARTE, ESTADO DO CONHECIMENTO)
%%
%% O autor deve registrar seu conhecimento sobre a literatura básica do assunto, discutindo e comentando a informação já publicada.
%% A revisão deve ser apresentada, preferencialmente, em ordem cronológica e por blocos de assunto, procurando mostrar a evolução do tema.
%% Título e rótulo de capítulo (rótulos não devem conter caracteres especiais, acentuados ou cedilha)
\chapter{Referencial teórico}
\label{cap:referencialTeorico}


\section{Fundamentação teórica}
\label{section:background}
Falar sobre \gls{sds} usando o \cite{DistributedSystemsCoulouris} como base

\subsection{Exemplos trabalhados}

% Definição de exemplo trabalhado
A definição de exemplo trabalhado (do inglês, \textit{worked example}) encontra-se claramente difundida na literatura, passível de ser compreendido como um trabalho cognitivo e experimental com o intuito de fornecer uma solução ideal, todavia próxima ao praticável, para um problema específico, no qual uma pessoa com escasso ou nenhum conhecimento acerca do tema possa examinar e aprender com a solução proposta. Sendo assim, o desenvolvimento e estudo de exemplos trabalhados enriquecem a qualidade do que é lecionado em sala de aula, uma vez que a solução ideal pode representar muito bem o estado da arte para um determinado tópico, visto que dissemina conceitos e padrões do problema apresentado \cite{Robert.Atkinson-etal:2000}.

Conforme exposto no artigo \textit{``Learning from Examples - Instructional Principles from the Worked Examples Research''} \cite{Robert.Atkinson-etal:2000}, os elementos essenciais na elaboração de um exemplo trabalhado de alta qualidade estão intrinsecamente vinculados aos conceitos de \textit{``inter-example feature''} e \textit{``intra-example feature''}. A concepção do termo \textit{``inter-example feature''} destaca a importância de fornecer uma diversidade de exemplos que ilustrem estratégias múltiplas para problemas similares, porém de tipos diferentes, enquanto o termo \textit{``intra-example feature''} insinua características intrínsecas à estrutura interna de cada exemplo. 

% Definição de exemplo trabalhado em educação em computação
O artigo intitulado ``Worked Examples in Computer Science'' \cite{Skudder-LuxtonReilly:2014} destaca a escassez de pesquisas sobre exemplos trabalhados na área da Ciência da Computação. Isso é evidenciado pelo fato de que apenas uma minoria das referências citadas no artigo aborda efetivamente exemplos trabalhados relacionados à Ciência da Computação, uma vez que as tarefas de programação requerem um alto teor cognitivo. Além disso, na presente pesquisa, o autor investiga as ramificações de exemplos trabalhados na área da Computação, frequentemente manifestados na apresentação passo a passo da execução de um código específico, prática recorrente em disciplinas como análise de algoritmos. Adicionalmente, o estudo explora o emprego do conceito de ``problem solution pair'', no qual um problema é apresentado, estimulando o aluno a tentar resolvê-lo, seguido pela exposição da solução correspondente. Por fim, de maneira menos convencional, são utilizadas técnicas de desenvolvimento de software enquanto se leciona um conteúdo, o que contribui para esclarecer o raciocínio empregado na elaboração da solução para o problema.

O impacto dos exemplos trabalhados na aprendizagem varia conforme o nível de experiência do aprendiz. Para iniciantes, esses exemplos proveem benefícios ao frisar a atenção nos elementos cruciais do problema, facilitando a criação de esquemas de resolução pertinentes e otimizando recursos cognitivos em comparação com a abordagem direta na resolução de problemas. Em contraste, para indivíduos com certa experiência, os exemplos trabalhados podem tornar-se supérfluos, resultando em uma carga cognitiva extrínseca \cite{Skudder-LuxtonReilly:2014}.

% Exemplos trabalhos para engenharia de software
No contexto contemporâneo, deparamo-nos com um desafio significativo associado à evolução incessante do conhecimento em domínios como Medicina, Física e, notavelmente, a Computação. Nesse contexto, disciplinas como (\gls{sd}) e (\gls{es}) apresentam desafios substanciais tanto para discentes quanto para docentes. Considerando a contribuição positiva do uso de exemplos trabalhados para a qualidade do ensino, torna-se imperativo analisar a perspectiva do educador. Pesquisas indicam que os pedagogos enfrentam obstáculos ao selecionar e desenvolver exemplos trabalhados, além de evidenciar a falta de pesquisa sobre exemplos trabalhados na \gls{es}. Acrescentando a isso, é importante ressaltar que apenas uma minoria dos professores faz uso de exemplos reais, apesar de quase a totalidade incorporar exemplos em suas práticas pedagógicas, conforme evidenciado por um \textit{survey} realizado no artigo intitulado ``Using Real Worked Examples to Aid Software Engineering Teaching'' \cite{Simone.Tonhao-etal:2021}.

Ademais, ao abordar a exposição dos alunos a projetos \textit{open-source}, um dos principais desafios reside na seleção de projetos apropriados. Idealmente, um projeto ativo com múltiplos contribuidores externos seria uma escolha favorável. No entanto, o maior desafio reside em engajar o estudante para realizar uma contribuição significativa, uma vez que projetos reais demandam um nível técnico considerável \cite{OSSProjects-TheProfessors'Perspective}. 


\cite{Simone.Tonhao-etal:2022}
\cite{Tiago.Bonetti-etal:2023}

\subsection{Métricas de software}

A Metrics Suite for Object-Oriented Design.
\cite{MetricsSuite}

\begin{itemize}
    \item \gls{cbo}: Consiste na definição do grau de acoplamento (dependências) que uma determinada classe apresenta. Quanto maior for o valor do \gls{cbo}, maior é o grau de acoplamento da classe, o que indica maior interdependência entre classes. Isso pode tornar o código mais complexo e menos flexível, uma vez que alterações na classe afetaria o comportamento de inúmeras outras classes. Dessa forma, para um determinado código, dada a sua evolução, quando observada uma queda nesse valor, tem-se então um bom indício de melhorias relevantes na qualidade daquele código.
    \item \gls{cbom}: Métrica similar a anterior, todavia considera também a dependência de classes uma referência a um objeto do tipo, ou seja, ao adicionar uma simples chamada de um método da classe, essa métrica é incrementada. Intuitivamente, o \gls{cbom} foi utilizado nos estágios iniciais do estudo foi retirado por poluir os valores, visto que o escopo de refatorações nas métricas foi menos restrito a refatorações da classe ou do método.
    \item \gls{rfc}: Refere-se ao número de invocações únicas de um método de uma determinada classe, isto é, a métrica conta o número de invocações estáticas. 
    OBS: Analisar e explicar o impacto dos valores dessa métrica.
    \item \gls{wmc}: OBS: Analisar e explicar o impacto dos valores dessa métrica de acordo com \cite{MetricsSuite}
\end{itemize}

\subsection{Educação em sistemas distribuídos}


\section{Trabalhos relacionados}
\label{section:related-work}

DistFax: A Toolkit for Measuring Interprocess Communications and Quality of Distributed Systems.
\cite{DistFax}
