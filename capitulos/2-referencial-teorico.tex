%%%% CAPÍTULO 2 - REVISÃO DA LITERATURA (OU REVISÃO BIBLIOGRÁFICA, ESTADO DA ARTE, ESTADO DO CONHECIMENTO)
%%
%% O autor deve registrar seu conhecimento sobre a literatura básica do assunto, discutindo e comentando a informação já publicada.
%% A revisão deve ser apresentada, preferencialmente, em ordem cronológica e por blocos de assunto, procurando mostrar a evolução do tema.
%% Título e rótulo de capítulo (rótulos não devem conter caracteres especiais, acentuados ou cedilha)
\chapter{Referencial teórico}
\label{cap:referencialTeorico}


\section{Fundamentação teórica}
\label{section:background}
Falar sobre SDs usando o colouris


\subsection{Exemplos trabalhados}


\subsection{Métricas de software}

A Metrics Suite for Object-Oriented Design.


\subsection{Educação em sistemas distribuídos}



\section{Trabalhos relacionados}
\label{section:related-work}

DistFax: A Toolkit for Measuring Interprocess Communications and Quality of Distributed Systems.
    
