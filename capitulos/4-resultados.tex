\chapter{Resultados}
\label{chapter:results}

% Seleção de Projetos.
\section{Seleção de projetos}

% Critérios de seleção de repositórios.
Os critérios para a seleção de repositórios foram determinados pelos seguintes fatores:

\begin{itemize}
    \item \textbf{Ser um repositório Java:} Essencial para viabilizar a aplicação das métricas \gls{ck}.
    \item \textbf{Ser de código aberto (\textit{open-source}):} A escolha por repositórios abertos, hospedados em plataformas como o \textit{Github}, visa garantir a transparência e acessibilidade do código-fonte.
    \item \textbf{Ser ativamente mantido e atualizado:} A ativa manutenção é crucial para garantir que os repositórios estejam alinhados com as práticas e avanços mais recentes.
    \item \textbf{Ser utilizado em aplicações do mundo real:} A seleção de repositórios empregados em contextos práticos proporciona uma análise mais relevante e aplicável às situações reais.
\end{itemize}

Dessa forma, os repositórios escolhidos foram o \textit{``Apache Kafka''} \cite{KafkaGitHub} e o \textit{``ZooKeeper''} \cite{ZookeeperGitHub}. Essa escolha baseou-se no fato de ambos os repositórios serem projetos \textit{open-source} disponíveis na plataforma GitHub, onde ambos são mantidos e atualizados de forma ativa para serem utilizados em ambientes do mundo real. Além disso, esses projetos lidam com questões altamente relevantes em \gls{sds}.

O \textit{``ZooKeeper''} desempenha um papel fundamental na coordenação e gerenciamento de serviços distribuídos. Ele oferece um serviço de consenso altamente confiável para \gls{sds}, garantindo a consistência e a sincronização entre os nós. Essa funcionalidade é crucial para a implementação de serviços distribuídos confiáveis e escaláveis, tornando o \textit{``ZooKeeper''} uma escolha valiosa para este estudo.

Por outro lado, o \textit{``Apache Kafka''} destaca-se no processamento de fluxos de dados distribuídos em larga escala. Ele fornece uma plataforma robusta para a transmissão eficiente de eventos entre diferentes componentes de um \gls{sd}. A capacidade do \textit{``Apache Kafka''} lidar com volumes massivos de dados em tempo real o torna uma solução amplamente adotada em ambientes que demandam alto desempenho e escalabilidade.

Dessa forma, o ambos os repositórios têm muito a agregar em termos de implementação e evolução tanto para a área de \gls{sds} quanto para a \gls{es}. Assim, a escolha desses repositórios permite a análise da evolução do código em contextos práticos e desafiadores, contribuindo para uma compreensão mais abrangente das práticas de desenvolvimento em \gls{sds}.

\section{Métricas Preliminarmente Selecionadas}
% Assim, no capítulo 4 podemos ter uma seção sobre as métricas preliminarmente selecionadas e, em seguida, uma seção sobre a ferramenta, automatizando a coleta, cálculo e visualização das métricas. Nesse sentido, podem ser apresentados também alguns resultados (figuras inclusive), sem entrar no mérito de análise.

% Desenvolvimento das ferramentas disponíveis no GitHub feitas em Python.
\section{Desenvolvimento da ferramenta}
As ferramentas desenvolvidas podem ser encontradas \href{https://github.com/BrenoFariasdaSilva/Scientific-Research}{aqui}, estando as três primeiras no diretório ``\textit{PyDriller}'' e as duas últimas no diretório ``\textit{RefactoringMiner}'', sendo que todas foram feitas usando a linguagem de programação \cite{PythonProgrammingLanguage}.

A \href{https://github.com/BrenoFariasdaSilva/Scientific-Research/blob/main/PyDriller/metrics_changes.py}{primeira ferramenta} desenvolvida está integrada ao \cite{PyDrillerSpadini2018} e ao \gls{ck}. Durante esse processo, a ferramenta percorre toda a árvore de \textit{commits} de um repositório específico. Para cada \textit{commit}, o \gls{ck} é executado, resultando na geração das métricas \gls{ck} para cada estado do repositório ao final da execução. Adicionalmente, a ferramenta produz um \textit{diff} para cada refatoração realizada no repositório, possibilitando a investigação das refatorações realizadas no código em caso de identificação de uma tendência de melhoria das métricas.

A \href{https://github.com/BrenoFariasdaSilva/Scientific-Research/blob/main/PyDriller/metrics_changes.py}{segunda ferramenta} desenvolvida depende da execução do primeiro código para operar. Essencialmente, esta ferramenta percorre as métricas de todos os estados do repositório processado, produzindo os seguintes resultados:

\begin{itemize}
    \item \textbf{Evolução das métricas}: Geração de arquivos divididos por classe e métodos, contendo o histórico completo de alterações em cada classe ou método, incluindo suas métricas correspondentes.
    \item \textbf{Predição das métricas}: Criação de arquivos segmentados por classe e métodos, apresentando uma predição baseada na regressão linear aplicada às métricas analisadas, utilizando o histórico gerado anteriormente.
    \item \textbf{Estatísticas das métricas}: Produção de um arquivo para classes e outro para métodos, ordenados pelo número de alterações efetuadas em cada classe ou método.
    \item \textbf{Alterações substanciais nas métricas}: Identificação de padrões de decréscimo do \gls{cbo}, oferecendo uma análise detalhada das classes ou métodos que apresentam tendências de melhoria ao longo do tempo.
\end{itemize}

Assim, resumidamente, esse código gera metadados que facilitam a análise dos códigos, reduzindo o escopo de busca por refatorações que, à luz das métricas, são consideradas relevantes para nosso estudo.

A \href{https://github.com/BrenoFariasdaSilva/Scientific-Research/blob/main/PyDriller/Scripts/track_files.py}{terceira ferramenta} desenvolvida, mais uma vez, depende da execução do primeiro código para funcionar. Em suma, esta ferramenta busca por arquivos específicos nos \textit{diffs} dos repositórios. Em nosso estudo, foi utilizada para localizar arquivos de comentários sobre as refatorações realizadas em um \textit{commit}, como o arquivo ``CHANGES.txt'', uma vez que o espaço da mensagem do \textit{commit} pode oferecer uma visão limitada das refatorações efetuadas.

A \href{https://github.com/BrenoFariasdaSilva/Scientific-Research/blob/main/PyDriller/Scripts/track_files.py}{quarta ferramenta} e a \href{https://github.com/BrenoFariasdaSilva/Scientific-Research/blob/main/RefactoringMiner/metrics_evolution_refactors.py}{quinta ferramenta} desenvolvidas são substancialmente similares, distinguindo-se apenas no escopo de aplicação, onde a quarta  incide sobre a totalidade do repositório, enquanto a quinta é específica para classes ou métodos dentro do repositório. Ambas as ferramentas compartilham um funcionamento comum, fazendo uso da ferramenta \cite{Tsantalis:ICSE:2018:RefactoringMiner}.

O processo inicia-se com a utilização da ferramenta \textit{RefactoringMiner}, a qual, por meio da extração do histórico de \textit{commits}, analisa os \textit{diffs}, identificando padrões de refatorações. O \textit{RefactoringMiner} possui um banco de dados de refatorações previamente detectadas, permitindo uma análise eficiente das mudanças no código-fonte. A ferramenta gera uma saída estruturada que lista as refatorações identificadas, indicando quais arquivos e linhas de código foram afetados em cada \textit{commit}.

Essa saída estruturada proporciona uma base sólida para análises mais aprofundadas, permitindo uma compreensão detalhada das refatorações realizadas ao longo do tempo. A capacidade de distinguir entre o repositório inteiro (quarta ferramenta) e componentes específicos, como classes ou métodos (quinta ferramenta), amplia a flexibilidade dessas ferramentas, adaptando-as às necessidades específicas de investigação e análise de refatorações em projetos de software.

% Análise das métricas dos projetos selecionados.
\section{Análise das métricas dos projetos selecionados}
Apesar da intenção inicial de analisar dois repositórios, o \textit{``Apache Kafka''} \cite{KafkaGitHub} e o \textit{``ZooKeeper''} \cite{ZookeeperGitHub}, até o presente momento, concentramos nossos esforços exclusivamente no exame do repositório do \textit{``ZooKeeper''}. Esta escolha se fundamenta no tamanho inferior deste repositório em comparação com o \textit{``Apache Kafka''}, permitindo uma análise mais focalizada e a possibilidade de apresentar resultados preliminares neste trabalho. A decisão foi motivada por considerações de limitação de tempo, pois a análise detalhada de um único repositório já representa um desafio significativo, demandando uma abordagem minuciosa para compreender a evolução do código-fonte e as refatorações realizadas.  Neste trabalho, iremos mencionar alguns casos analisados que aparentavam ser promissores, no entanto, mostraram algumas dificuldades a mais que iremos lidar.

A análise atual revela a presença de vários \textit{commits} no histórico do repositório do \textit{``ZooKeeper''}, e notamos uma tendência decrescente no valor de métricas, como o \gls{cbo}. No entanto, é crucial observar que apenas uma parcela restrita desses \textit{commits} inclui mensagens associadas a melhorias de desempenho, segurança, substituição de algoritmos ou melhorias de modo geral. 

Lamentavelmente, até o momento, apenas um \textit{commit} apresentou uma correlação direta entre as vulnerabilidades identificadas no banco de dados do \textit{``CVE-Details''} e os \textit{commits} no repositório do \textit{``ZooKeeper''}. O único caso encontrado foi para o \textit{commit} 9213f7353b1e6ce4d0fdbc1dca963ace1fd32cec, o qual relaciona-se a vulnerabilidade CVE-2021-21295. Entretando, por mais que o \textit{commit} resolva um problema de vulnerabilidade, constatamos que esta refatoração específica consiste apenas na atualização da versão da biblioteca \textit{Netty} no arquivo pom.xml do Java, o qual prove uma estrutura cliente-servidor de E/S não bloqueante, logo ela não se mostra uma boa candidata para correlacionar métricas de código com a segurança, pois se trata de uma ação direta para resolver uma vulnerabilidade conhecida.

Aprofundamos a análise do \textit{commit} 83cf0a93c37759334fab885c2010fa0b7d953f52, cuja mensagem é "\textit{ZOOKEEPER-308. Improve the atomic broadcast performance in 3x}". A princípio, esse \textit{commit} parecia promissor, pois indicava um impressionante ganho de desempenho de 3 vezes. No entanto, após uma investigação mais detalhada, constatamos que a refatoração se resumia à substituição da biblioteca \textit{"FileOutputStream"} por \textit{"BufferedOutputStream"}. Essa mudança, embora resulte em uma melhoria de desempenho, não se configura como uma refatoração significativa, uma vez que o \textit{commit} altera 19 arquivos, mas apenas o arquivo "\textit{FileTxnLog.java}" é relevante, pois consiste na troca das bibliotecas mencionadas. Os outros arquivos alterados, infelizmente, fogem do escopo da mensagem do \textit{commit}, visto que as refatorações estariam relacionadas com a validação se um objeto é nulo, antes de realizar alguma operação com o mesmo. No entendo, admitimos a melhoria de desempenho como válida, embora sua magnitude não possa ser verificada.

A falta de uma refatoração mais substancial e a existência de uma melhoria de desempenho que não pôde ser validada tornam este \textit{commit} inadequado para servir como objeto de exemplo trabalhado. Além disso, uma vez que a classe \textit{"BufferedOutputStream"} está presente desde a versão inicial do Java (lançada em 1996) e o \textit{commit} em questão é de 2009, ou seja, o uso de um \textit{buffered input} devia ter sido feita desde o começo do projeto.

% Adicionar pelo menos mais um exemplo de commit analisado.

Observamos que parte da complexidade na análise de um \textit{commit} propriamente dito reside na existência de uma considerável poluição nos dados. Muitos \textit{commits} no repositório do \textit{``ZooKeeper''} abrangem alterações em inúmeros arquivos, frequentemente incorporando refatorações não diretamente correlacionadas às mensagens de \textit{commit}. Essa complexidade aumenta a dificuldade de discernir as motivações subjacentes por trás das mudanças no código-fonte, tornando desafiador o processo de vincular essas alterações a possíveis vulnerabilidades identificadas externamente. Este fenômeno destaca a importância de uma abordagem criteriosa na interpretação do histórico de \textit{commits} para extrair conclusões significativas sobre a evolução do \textit{software} em relação à qualquer aspecto desejado.

\section{Limitações}
\label{sec:limitacoes}

% Limitações do projeto de pesquisa.
Este trabalho apresenta uma limitação, a qual reside na impossibilidade de realizar análises empíricas de certas melhorias de desempenho, como a análise das trocas de mensagens em tempo real, devido à restrição ao uso de métricas estáticas de código. Isso impede a avaliação prática por meio da execução de \gls{sds} e simulações em grande escala, representando uma restrição na obtenção de percepções detalhadas sobre o desempenho em contextos de uma quantidade massiva de requisições sendo enviadas ao \gls{sd}.