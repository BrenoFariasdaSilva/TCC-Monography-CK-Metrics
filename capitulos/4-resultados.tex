\chapter{Resultados}
\label{chapter:results}

% .
\section{Seleção de projetos}

Os critérios para a seleção de repositórios foram determinados pelos seguintes fatores:
\begin{itemize}
    \item Ser um repositório Java, viabilizando a aplicação das métricas \gls{ck};
    \item O repositório deve ser de código aberto (\textit{open-source}), hospedado em plataformas como o \textit{Github};
    \item Ser um repositório ativamente mantido e atualizado;
    \item Ser um repositório utilizado em aplicações do mundo real;  
\end{itemize}

% Dessa forma, os repositórios escolhidos foram ..;, pelas seguintes razões ...

% .
\section{Desenvolvimento da ferramenta}


% .
\section{Análise das métricas dos projetos selecionados}
% Mencionar o caso de melhorias de performance do ZooKeeper, onde o commit altera inúmeros arquivos mas em si é apenas uma atualização da biblioteca, a qual existe desde o java 1.0 que foi lançado em torno de 1996.

\section{Limitações}
\label{sec:limitacoes}
Este trabalho apresenta uma limitação, a qual reside na impossibilidade de realizar análises empíricas de certas melhorias de desempenho, como a análise das trocas de mensagens em tempo real, devido à restrição ao uso de métricas estáticas de código. Isso impede a avaliação prática por meio da execução de \gls{sds} e simulações em grande escala, representando uma restrição na obtenção de percepções detalhadas sobre o desempenho em contextos de uma quantidade massiva de requisições sendo enviadas ao \gls{sd}.
