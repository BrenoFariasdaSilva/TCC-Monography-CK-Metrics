\chapter{Resultados}
\label{chapter:results}

% Seleção de Projetos.
\section{Seleção de projetos}

% Critérios de seleção de repositórios.
Os critérios para a seleção de repositórios foram determinados pelos seguintes fatores:

\begin{itemize}
    \item \textbf{Ser um repositório Java:} Essencial para viabilizar a aplicação das métricas \gls{ck}.
    \item \textbf{Ser de código aberto (\textit{open-source}):} A escolha por repositórios abertos, hospedados em plataformas como o \textit{Github}, visa garantir a transparência e acessibilidade do código-fonte.
    \item \textbf{Ser ativamente mantido e atualizado:} A ativa manutenção é crucial para garantir que os repositórios estejam alinhados com as práticas e avanços mais recentes.
    \item \textbf{Ser utilizado em aplicações do mundo real:} A seleção de repositórios empregados em contextos práticos proporciona uma análise mais relevante e aplicável às situações reais.
\end{itemize}

Dessa forma, os repositórios escolhidos foram o \textit{``Apache Kafka''} \cite{KafkaGitHub} e o \textit{``ZooKeeper''} \cite{ZookeeperGitHub}. Essa escolha baseou-se no fato de ambos os repositórios serem projetos \textit{open-source} disponíveis na plataforma GitHub, onde ambos são mantidos e atualizados de forma ativa para serem utilizados em ambientes do mundo real. Além disso, esses projetos lidam com questões altamente relevantes em \gls{sds}.

O \textit{``ZooKeeper''} desempenha um papel fundamental na coordenação e gerenciamento de serviços distribuídos. Ele oferece um serviço de consenso altamente confiável para \gls{sds}, garantindo a consistência e a sincronização entre os nós. Essa funcionalidade é crucial para a implementação de serviços distribuídos confiáveis e escaláveis, tornando o \textit{``ZooKeeper''} uma escolha valiosa para este estudo.

Por outro lado, o \textit{``Apache Kafka''} destaca-se no processamento de fluxos de dados distribuídos em larga escala. Ele fornece uma plataforma robusta para a transmissão eficiente de eventos entre diferentes componentes de um \gls{sd}. A capacidade do \textit{``Apache Kafka''} lidar com volumes massivos de dados em tempo real o torna uma solução amplamente adotada em ambientes que demandam alto desempenho e escalabilidade.

Dessa forma, o ambos os repositórios têm muito a agregar em termos de implementação e evolução tanto para a área de \gls{sds} quanto para a \gls{es}. Assim, a escolha desses repositórios permite a análise da evolução do código em contextos práticos e desafiadores, contribuindo para uma compreensão mais abrangente das práticas de desenvolvimento em \gls{sds}.

% Desenvolvimento das ferramentas disponíveis no GitHub feitas em Python.
\section{Desenvolvimento da ferramenta}
As ferramentas desenvolvidas podem ser encontradas \href{https://github.com/BrenoFariasdaSilva/Scientific-Research}{aqui}, estando as três primeiras no diretório ``\textit{PyDriller}'' e as duas últimas no diretório ``\textit{RefactoringMiner}'', sendo que todas foram feitas usando a linguagem de programação \cite{PythonProgrammingLanguage}.

A \href{https://github.com/BrenoFariasdaSilva/Scientific-Research/blob/main/PyDriller/metrics_changes.py}{primeira ferramenta} desenvolvida está integrada ao \cite{PyDrillerSpadini2018} e ao \gls{ck}. Durante esse processo, a ferramenta percorre toda a árvore de \textit{commits} de um repositório específico. Para cada \textit{commit}, o \gls{ck} é executado, resultando na geração das métricas \gls{ck} para cada estado do repositório ao final da execução. Adicionalmente, a ferramenta produz um \textit{diff} para cada refatoração realizada no repositório, possibilitando a investigação das refatorações realizadas no código em caso de identificação de uma tendência de melhoria das métricas.

A \href{https://github.com/BrenoFariasdaSilva/Scientific-Research/blob/main/PyDriller/metrics_changes.py}{segunda ferramenta} desenvolvida depende da execução do primeiro código para operar. Essencialmente, esta ferramenta percorre as métricas de todos os estados do repositório processado, produzindo os seguintes resultados:

\begin{itemize}
    \item \textbf{Evolução das métricas}: Geração de arquivos divididos por classe e métodos, contendo o histórico completo de alterações em cada classe ou método, incluindo suas métricas correspondentes.
    \item \textbf{Predição das métricas}: Criação de arquivos segmentados por classe e métodos, apresentando uma predição baseada na regressão linear aplicada às métricas analisadas, utilizando o histórico gerado anteriormente.
    \item \textbf{Estatísticas das métricas}: Produção de um arquivo para classes e outro para métodos, ordenados pelo número de alterações efetuadas em cada classe ou método.
    \item \textbf{Alterações substanciais nas métricas}: Geração de arquivos para classes e métodos, ordenados pela queda percentual dos valores da métrica \gls{cbo} em cada classe ou método.
\end{itemize}

Assim, resumidamente, esse código gera metadados que facilitam a análise dos códigos, reduzindo o escopo de busca por refatorações que, à luz das métricas, são consideradas relevantes para nosso estudo.

A \href{https://github.com/BrenoFariasdaSilva/Scientific-Research/blob/main/PyDriller/Scripts/track_files.py}{terceira ferramenta} desenvolvida, mais uma vez, depende da execução do primeiro código para funcionar. Em suma, esta ferramenta busca por arquivos específicos nos \textit{diffs} dos repositórios. Em nosso estudo, foi utilizada para localizar arquivos de comentários sobre as refatorações realizadas em um \textit{commit}, como o arquivo ``CHANGES.txt'', uma vez que o espaço da mensagem do commit pode oferecer uma visão limitada das refatorações efetuadas.

A \href{https://github.com/BrenoFariasdaSilva/Scientific-Research/blob/main/PyDriller/Scripts/track_files.py}{quarta ferramenta} e a \href{https://github.com/BrenoFariasdaSilva/Scientific-Research/blob/main/RefactoringMiner/metrics_evolution_refactors.py}{quinta ferramenta} desenvolvidas são substancialmente similares, distinguindo-se apenas no escopo de aplicação, onde a quarta  incide sobre a totalidade do repositório, enquanto a quinta é específica para classes ou métodos dentro do repositório. Ambas as ferramentas compartilham um funcionamento comum, fazendo uso da ferramenta \cite{Tsantalis:ICSE:2018:RefactoringMiner}.

O processo inicia-se com a utilização da ferramenta \textit{RefactoringMiner}, a qual, por meio da extração do histórico de \textit{commits}, analisa os \textit{diffs}, identificando padrões de refatorações. O \textit{RefactoringMiner} possui um banco de dados de refatorações previamente detectadas, permitindo uma análise eficiente das mudanças no código-fonte. A ferramenta gera uma saída estruturada que lista as refatorações identificadas, indicando quais arquivos e linhas de código foram afetados em cada \textit{commit}.

Essa saída estruturada proporciona uma base sólida para análises mais aprofundadas, permitindo uma compreensão detalhada das refatorações realizadas ao longo do tempo. A capacidade de distinguir entre o repositório inteiro (quarta ferramenta) e componentes específicos, como classes ou métodos (quinta ferramenta), amplia a flexibilidade dessas ferramentas, adaptando-as às necessidades específicas de investigação e análise de refatorações em projetos de software.

% Análise das métricas dos projetos selecionados.
\section{Análise das métricas dos projetos selecionados}
Apesar da intenção inicial de analisar dois repositórios, o \textit{``Apache Kafka''} \cite{KafkaGitHub} e o \textit{``ZooKeeper''} \cite{ZookeeperGitHub}, até o presente momento, concentramos nossos esforços exclusivamente no exame do repositório do \textit{``ZooKeeper''}. Essa decisão foi motivada por considerações de limitação de tempo, uma vez que a análise detalhada de um único repositório já apresenta desafios significativos e demanda uma abordagem minuciosa para a compreensão profunda da evolução do código-fonte e das refatorações realizadas.

A análise atual revela a presença de vários \textit{commits} no histórico do repositório do \textit{``ZooKeeper''}, e notamos uma tendência decrescente no valor de métricas, como o \gls{cbo}. No entanto, é crucial observar que apenas uma parcela restrita desses \textit{commits} inclui mensagens associadas a melhorias de desempenho, segurança, substituição de algoritmos ou melhorias em geral. Lamentavelmente, até o momento, não conseguimos estabelecer uma correlação direta entre as vulnerabilidades identificadas no banco de dados do \textit{``CVE-Details''} e os \textit{commits} no repositório do \textit{``ZooKeeper''}. O único caso para o commit 9213f7353b1e6ce4d0fdbc1dca963ace1fd32cec relacionado a uma vulnerabilidade (CVE-2021-21295), indicando a atualização da biblioteca Netty para a versão 4.1.60 devido à mencionada vulnerabilidade de segurança. Após uma inspeção mais detalhada, constatamos que esta refatoração específica, que consiste na atualização da versão do Netty no arquivo pom.xml do Java, não se mostra uma boa candidata para correlacionar métricas de código com a segurança, pois se trata de uma ação direta para resolver uma vulnerabilidade conhecida.

Observamos que parte da complexidade na análise de um \textit{commit} propriamente dito reside na existência de uma considerável poluição nos dados. Muitos \textit{commits} no repositório do \textit{``ZooKeeper''} abrangem alterações em inúmeros arquivos, frequentemente incorporando refatorações não diretamente correlacionadas às mensagens de \textit{commit}. Essa complexidade aumenta a dificuldade de discernir as motivações subjacentes por trás das mudanças no código-fonte, tornando desafiador o processo de vincular essas alterações a possíveis vulnerabilidades identificadas externamente. Este fenômeno destaca a importância de uma abordagem criteriosa na interpretação do histórico de \textit{commits} para extrair conclusões significativas sobre a evolução do \textit{software} em relação à qualquer aspecto desejado.

% Mencionar o caso da atualização da biblioteca.

\section{Limitações}
\label{sec:limitacoes}

% Limitações do projeto de pesquisa.
Este trabalho apresenta uma limitação, a qual reside na impossibilidade de realizar análises empíricas de certas melhorias de desempenho, como a análise das trocas de mensagens em tempo real, devido à restrição ao uso de métricas estáticas de código. Isso impede a avaliação prática por meio da execução de \gls{sds} e simulações em grande escala, representando uma restrição na obtenção de percepções detalhadas sobre o desempenho em contextos de uma quantidade massiva de requisições sendo enviadas ao \gls{sd}.