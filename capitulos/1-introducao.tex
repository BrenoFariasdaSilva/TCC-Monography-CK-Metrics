%%%% CAPÍTULO 1 - INTRODUÇÃO
%%
%% Deve apresentar uma visão global da pesquisa, incluindo: breve histórico, importância e justificativa da escolha do tema,
%% delimitações do assunto, formulação de hipóteses e objetivos da pesquisa e estrutura do trabalho.

%% Título e rótulo de capítulo (rótulos não devem conter caracteres especiais, acentuados ou cedilha)

\chapter{Introdução}\label{cap:introducao}

% Contexto Sistemas Distribuídos.
No cenário atual da computação, os \gls{sds} desempenham um papel fundamental, alimentando a infraestrutura de serviços e aplicações que impulsionam nosso mundo digital. De acordo com \citeonline[p. 968]{TanenbaumDistributedSystemsThirdEdition}, um \gls{sd} é definido como ``Um conjunto de computadores independentes que se apresenta a seus usuários como um sistema único e coerente.'' Diante disso, emergem desafios inerentes devido à necessidade de apresentar o sistema ao usuário como uma entidade homogênea, embora, intrinsecamente, o sistema seja constituído por diversas partes heterogêneas.

% Uso de Sistemas Distribuídos no mundo.
A complexidade inerente aos \gls{sds} os torna um campo de pesquisa desafiador, mas, ao mesmo tempo, altamente relevante na computação moderna. Compreender e lidar com essa heterogeneidade e complexidade é essencial para o desenvolvimento, manutenção e escalabilidade de \gls{sds}. De igual modo, a crescente dependência de empresas e instituições por esses sistemas torna o estudo de \gls{sds} indispensável para a formação de profissionais capacitados e para o avanço da qualidade do ensino em Engenharia de Sofware (\gls{es}).

% Preocupações em Sistemas Distribuídos.
Nos \gls{sds}, as preocupações com a segurança dos dados e da comunicação, o desempenho eficiente e a capacidade de manter a operação contínua, mesmo diante de falhas, são instigações extremamente pertinentes. A necessidade de proteger informações sensíveis, assegurar rápido tempo de resposta, garantir que a troca de mensagens seja eficiente e manter a disponibilidade de serviços torna o estudo da evolução do código em \gls{sds} ainda mais vital. Essas complexidades são estudadas em cursos universitários e estão intrinsecamente ligadas ao código que impulsiona esses sistemas, tornando a investigação sobre a evolução do código em \gls{sds} uma fonte valiosa sobre como esses sistemas evoluem para atender às demandas constantes.

% Educação em Sistemas Distribuídos.
Os currículos de Ciência da Computação ACM CS2023 \cite{CS2023:ACM} não abordam estudos específicos sobre \gls{sds} no contexto educacional. Um estudo identificou que tópicos como Processos e Threads, Replicação, Chamadas de Sistema, Controle de Concorrência, Tolerância a Falhas, Sincronização, Comunicação, entre outros constituem os elementos-chave em cursos de \gls{sds} \cite{HaveWeReachedConsensus}. No entanto, a carência de pesquisas recentes salienta a importância de investigar a evolução do código em \gls{sds}, especialmente com ênfase na educação. Isto é evidenciado pela relevância crítica desses sistemas, explicada pelo avanço de tecnologias de software e hardware, pelo aumento do acesso à internet e número de usuários.

% Exemplos trabalhados em Ciência da Computação.
A eficácia do ensino em Ciência da Computação, particularmente no contexto de \gls{es}, está intrinsecamente ligada ao desenvolvimento e uso de exemplos trabalhados (do inglês, \textit{worked examples}). Um exemplo trabalhado é uma ferramenta pedagógica poderosa, definida como um trabalho cognitivo e experimental que oferece uma solução ideal e praticável para um problema específico, permitindo que aprendizes examinem e aprendam com a solução proposta \cite{Robert.Atkinson-etal:2000}. A literatura destaca a importância dos exemplos trabalhados na disseminação de conceitos e padrões, fornecendo uma solução representativa do estado da arte para um tópico específico. A falta de pesquisa dedicada aos exemplos trabalhados nesta área é evidente, e muitos professores enfrentam obstáculos ao incorporar exemplos reais em suas práticas pedagógicas \cite{Simone.Tonhao-etal:2021}.

% Exemplos trabalhados na Engenharia de Software.
No âmbito da \gls{es}, os exemplos trabalhados frequentemente assumem a forma de apresentação passo-a-passo da execução de códigos específicos. No entanto, pesquisas indicam a escassez de estudos específicos sobre exemplos trabalhados na Ciência da Computação, uma vez que tarefas de programação exigem alto teor cognitivo \cite{Skudder-LuxtonReilly:2014}. Além disso, estratégias como ``\textit{Faded Worked Examples}'', que envolvem a apresentação gradual de exemplos resolvidos, têm mostrado impacto significativo na aprendizagem, promovendo a abstração do aluno e desenvolvendo a capacidade de recuperar informações \cite{Skudder-LuxtonReilly:2014}.

\section{Objetivos}\label{sec:objetivos}
\subsection{Objetivo geral}\label{subsec:objetivoGeral}
% Objetivo Geral.
Neste contexto, o objetivo geral deste projeto de pesquisa é desenvolver uma heurística para identificar exemplos de código-fonte representativos do aprimoramento do \textit{software} de um \gls{sd} por meio da análise de métricas de código usadas na \gls{es}, de modo a criar um exemplo trabalhado para servir como material a ser utilizado em sala de aula. Essa heurística será projetada para auxiliar na seleção de exemplos trabalhados que demonstrem como o código em \gls{sds} se adaptam e evoluem ao longo do tempo. Em decorrência disso, uma vez conquistada a heurística almejada, objetiva-se criar um exemplo trabalhado, de modo a compreender por que um código específico atende às métricas de código desejadas no âmbito do desenvolvimento de software.

\subsection{Objetivos específicos}\label{subsec:objetivosEspecificos}
% Objetivos Específicos.
\begin{itemize}
    \item Desenvolver uma heurística de seleção que possa analisar as métricas de código.
    \item Identificar exemplos de código-fonte que representem eficazmente a evolução em \gls{sds} ao longo do tempo.
    \item Fornecer um exemplo trabalhado para a \gls{es}, identificando em quais aspectos um determinado código evoluiu, reconhecendo padrões, desafios comuns e práticas recomendadas.
\end{itemize}

\section{Justificativa}\label{sec:justificativa}
% Motivo de usar Sistemas Distribuídos como objeto de estudo.
Os avanços da digitalização de serviços e produtos tornou a sociedade dependente de \gls{sds}, acessados por usuários a partir de múltiplos dispositivos como computadores pessoais, smartphones, TVs e muito mais. A dependência abrange desde redes sociais até sistemas de transporte, saúde, serviços governamentais e muito mais. Portanto, para conquistar uma boa formação de estudantes e profissionais na área da Computação, faz-se necessário o entendimento de \gls{es} e \gls{sds}, tornando-se uma necessidade iminente para atender à demanda crescente por profissionais qualificados nesse campo. 

Dado o contexto anterior, a investigação sobre a evolução do código em \gls{sds} assume destaque. O estudo visa proporcionar uma compreensão mais profunda de como o código-fonte em \gls{sds} se adapta ao longo do tempo para atender a demandas como o aumento do número de usuários em escala não esperada, descobertas de vulnerabilidades e novas formas de ataque aos sistemas, além do uso de novos protocolos de comunicação para tentar resolver desafios enfrentados por estes sistemas.

% Objetivos do trabalho.
Os objetivos traçados neste trabalho se alinham com a necessidade de abordar essas complexidades. O desenvolvimento de uma heurística para identificar exemplos de código-fonte representativos da evolução em \gls{sds} contribuirá para que estudantes e profissionais possam entender melhor quais as melhores práticas e técnicas a serem utilizadas no desenvolvimento de tais sistemas, levando-se em consideração as inúmeras características de \gls{sds} já mencionadas anteriormente. Esses exemplos servirão como valiosos recursos de ensino e estudo, facilitando a compreensão das complexidades e desafios dos \gls{sds} em um ambiente distribuído e interconectado.

% Justificativa Geral.
Portanto, a justificativa para este estudo se baseia na importância crítica de compreender a evolução do código em \gls{sds}, contribuindo para a formação de profissionais mais capacitados e avançando a qualidade do ensino em \gls{es}. Este estudo procura preencher uma lacuna na literatura sobre a criação de exemplos trabalhados em \gls{sds} para ser objeto de estudo na \gls{es}, fornecendo percepções valiosas para o desenvolvimento e manutenção de \gls{sds} e aprimorando a prática profissional neste campo essencial da computação.

\section{Estrutura do trabalho}
\label{sec:estruturaTrabalho}

% Estrutura do trabalho.
A estrutura desta monografia segue uma abordagem organizada, dividida em cinco capítulos para uma compreensão abrangente do trabalho. O Capítulo \ref{cap:referencialTeorico} abrange a fundamentação teórica e revisão de trabalhos relacionados, proporcionando a base conceitual necessária para a compreensão do trabalho. O Capítulo \ref{cap:Metodologia} detalha a abordagem proposta, incluindo questões da pesquisa, ferramentas, métodos e um cronograma para orientar a condução da pesquisa. O Capítulo \ref{cap:resultados} apresenta os resultados preliminares, a partir da análise das métricas obtidas dos projetos selecionados. Finalmente, o Capítulo \ref{cap:conclusoes} oferece uma síntese dos principais resultados, mencionando as limitações do estudo.
