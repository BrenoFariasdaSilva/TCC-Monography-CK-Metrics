%%%% CAPÍTULO 1 - INTRODUÇÃO
%%
%% Deve apresentar uma visão global da pesquisa, incluindo: breve histórico, importância e justificativa da escolha do tema,
%% delimitações do assunto, formulação de hipóteses e objetivos da pesquisa e estrutura do trabalho.

%% Título e rótulo de capítulo (rótulos não devem conter caracteres especiais, acentuados ou cedilha)
\chapter{Introdução}\label{cap:introducao}
OBS: Remover sas explicaçoes dos capítulos.

\textbf{}Contextualizar: Motivação. Falar também o que é um SD, a importância de um SD no mundo atual, a importância do estudo de SDs". Explicar que o estudo sobre SDs pode servir para melhorar a qualidade do ensino e coisas do tipo. Usar Coulouris e tanenbaum como base/referência.

\section{Objetivos}\label{sec:objetivos}

\subsection{Objetivo geral}\label{subsec:objetivoGeral}

Desenvolver uma heurística para identificar exemplos relevantes de evolução de código por meio da análise de métricas CK. Isso nos permitirá selecionar exemplos trabalhados de qualidade para uso em sala de aula, proporcionando uma visão da evolução de software na área de Sistemas Distribuídos dispostos no mercado de trabalho.

ou

Desenvolver uma heurística para identificar exemplos trabalhados pertinentes com relação a evolução de código por meio da análise de métricas CK. Dessa forma, uma vez conquistada uma heurística, pretende-se escolher um bom exemplo trabalhado de modo a servir como material a ser utilizado em sala de aula , o qual irá proporcionar uma análise de como, no mercado de trabalho, softwares focados na área de Sistemas Distribuídos evoluem ao longo do tempo.

\subsection{Objetivos específicos}\label{subsec:objetivosEspecificos}
OBS: Colocar no infinitivo. Não é passo a passo, mas subprodutos que, quando completados, cumprem o objetivo geral
\begin{itemize}
    \item Escolher repositório(s) open-source apropriado(s) relacionado(s) à area de Sistemas Distribuídos.
    \item Seleção de ferramentas para geração de métricas do(s) repositório(s) selecionado(s).
    \item Seleção das métricas relevantes para a heurística a ser desenvolvida.
    \item Análise dos dados para validação da heurística.
    \item Refinamento da heurística.
    \item Seleção de possíveis bons exemplos e estudo do que é possível retirar de boas refatoraçôes.
\end{itemize}

\section{Justificativa}\label{sec:justificativa}
OBS: Justificativa em cima do seu objetivo.
Explicar que o resultado vai servir para ser usado em sala de aula para contrinbuir para o enriquecimento da universidade.

\section{Estrutura do trabalho}\label{sec:estruturaTrabalho}

OBS: Em algum canto na introdução, discorrer sobre a viabilidade do estudo e dos fatores que podem "inviabilizar" ela como, por exemplo, os dados serem muito poluídos no geral, coisa que é sim