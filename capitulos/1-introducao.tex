%%%% CAPÍTULO 1 - INTRODUÇÃO
%%
%% Deve apresentar uma visão global da pesquisa, incluindo: breve histórico, importância e justificativa da escolha do tema,
%% delimitações do assunto, formulação de hipóteses e objetivos da pesquisa e estrutura do trabalho.

%% Título e rótulo de capítulo (rótulos não devem conter caracteres especiais, acentuados ou cedilha)
\chapter{Introdução}\label{cap:introducao}
No cenário atual da computação, os \gls{sds} desempenham um papel fundamental, alimentando a infraestrutura de serviços e aplicações que impulsionam nosso mundo digital. De acordo com \citeonline[p. 968]{TanenbaumDistributedSystemsThirdEdition}, um \gls{sd} é definido como ``Um conjunto de computadores independentes que se apresenta a seus usuários como um sistema único e coerente''. Diante disso, emergem desafios inerentes devido à necessidade de apresentar o sistema ao usuário como uma entidade homogênea, embora, intrinsecamente, o sistema seja constituído por diversas partes heterogêneas. A heterogeneidade abrange diversos aspectos, incluindo variações em termos de rede, hardware, sistemas operacionais, linguagens de programação e implementações realizadas por diferentes desenvolvedores \cite{DistributedSystemsCoulouris}.

A complexidade inerente aos \gls{sds} os torna um campo desafiador, mas, ao mesmo tempo, altamente relevante na computação moderna. Compreender e lidar com essa heterogeneidade e complexidade é essencial para o desenvolvimento, manutenção e escalabilidade de \gls{sds}. De igual modo, a crescente dependência desses sistemas em nossa sociedade digital torna o estudo de \gls{sds} fundamental para a formação de profissionais capacitados e para o avanço da qualidade do ensino em \gls{es}.

Além disso, nos \gls{sds}, as preocupações com a segurança dos dados e da comunicação, o desempenho eficiente e a capacidade de manter a operação contínua, mesmo diante de falhas, são instigações extremamente pertinentes. A necessidade de proteger informações sensíveis, garantir tempos de resposta ágeis, garantir que a troca de mensagens seja eficiente e manter a disponibilidade de serviços torna o estudo da evolução do código em \gls{sds} ainda mais vital, uma vez que essas complexidades estão inerentemente ligadas ao código que impulsiona esses sistemas. É nesse contexto que a investigação sobre a evolução do código em \gls{sd} se destaca, proporcionando uma visão valiosa sobre como esses sistemas complexos evoluem para atender às demandas em constante mudança da era digital.

% TODO
No cenário atual, a literatura não oferece estudos suficientes sobre educação em Sistemas Distribuídos. Uma análise do estudo intitulado ``Have We Reached Consensus? An Analysis of Distributed Systems Syllabi'' \cite{HaveWeReachedConsensus} destaca a falta de pesquisas atualizadas nesse campo. Esse estudo identificou que tópicos como Processos e Threads, Replicação, Chamadas de Sistema, Controle de Concorrência, Tolerância a Falhas, Sincronização, Comunicação constituem os elementos-chave em cursos de \gls{sds}. No entanto, a carência de pesquisas recentes salienta a importância de investigar a evolução do código em \gls{sds}, especialmente com ênfase na educação, dada a relevância crítica desses sistemas na infraestrutura digital contemporânea.

% TODO
Exemplos de trabalho de educação em SD e afins: https://ieeexplore.ieee.org/document/10132035

% TODO
Exemplo trabalhado

\section{Objetivos}\label{sec:objetivos}
\subsection{Objetivo geral}\label{subsec:objetivoGeral}
 Neste contexto, o objetivo geral deste projeto de pesquisa é desenvolver uma heurística para identificar exemplos de código-fonte representativos do aprimoramento do \textit{software} de um \gls{sd} por meio da análise de métricas de código usadas na \gls{es}. Essa heurística será projetada para auxiliar na seleção de exemplos trabalhados que demonstrem como o código em \gls{sds} se adaptam e evoluem ao longo do tempo. Em decorrência disso, uma vez conquistada a heurística almejada, objetiva-se criar um exemplo trabalhado, de modo a compreender por que um código específico atende às métricas de código desejadas no âmbito do desenvolvimento de software.

\subsection{Objetivos específicos}\label{subsec:objetivosEspecificos}
\begin{itemize}
    \item .
    \item Desenvolver uma heurística de seleção que possa analisar as métricas de código e identificar exemplos de código-fonte que representem eficazmente a evolução em \gls{sds} ao longo do tempo.
    \item Identificar a análise da evolução de código realizada nos exemplos escolhidos, identificando padrões, desafios comuns e práticas recomendadas.
\end{itemize}

\section{Justificativa}\label{sec:justificativa}
A complexidade e heterogeneidade inerentes aos \gls{sds} estabelecem desafios significativos para o desenvolvimento, manutenção e escalabilidade desses sistemas. Compreender e abordar essas complexidades é crucial, dada a dependência crescente dos \gls{sds} em nossa sociedade digital. A dependência abrange desde redes sociais até sistemas de transporte, saúde e muito mais. Portanto, para conquistar uma boa formação de estudantes e profissionais na área da Computação, faz-se necessário o entendimento de \gls{es} e \gls{sds}, tornando-se uma necessidade iminente para atender à demanda crescente por profissionais qualificados nesse campo.
É nesse contexto que a investigação sobre a evolução do código em \gls{sds} assume destaque. O estudo visa proporcionar uma compreensão mais profunda de como o código-fonte em \gls{sds} se adapta ao longo do tempo para atender às demandas da era digital em constante evolução.

Os objetivos traçados neste trabalho se alinham com a necessidade de abordar essas complexidades. O desenvolvimento de uma heurística para identificar exemplos de código-fonte representativos da evolução em \gls{sds} contribuirá para a formação de estudantes e profissionais em \gls{es}. Esses exemplos servirão como valiosos recursos de ensino e estudo, facilitando a compreensão das complexidades e desafios dos \gls{sds} em um ambiente distribuído e interconectado.

Portanto, a justificativa para este estudo se baseia na importância crítica de compreender a evolução do código em \gls{sds}, contribuindo para a formação de profissionais mais capacitados e avançando a qualidade do ensino em \gls{es}. Este estudo procura preencher uma lacuna na literatura, fornecendo \textit{insights} valiosos para o desenvolvimento e manutenção de \gls{sds} e aprimorando a prática profissional neste campo essencial da computação.

\begin{itemize}
    \item OBS: Referenciar os trabalhados relacionados, identificando/expondo essa lacuna. "Os trabalhados atuais, tais como ..., até onde se sabe, não foram encontrados trabalhados que tentaram relacionar métricas estáticas com evolução de código")
\end{itemize}

Em projetos de desenvolvimento de software, a qualidade e organização dos dados desempenham um papel crucial na condução de pesquisas significativas. No entanto, um dos fatores que podem inviabilizar a pesquisa seria enfrentar com a presença de dados excessivamente poluídos. Isso pode ocorrer quando os \textit{commits} de código, responsáveis por registrar alterações no software, não são adequadamente divididos e agregam uma quantidade substancial de refatorações em um único \textit{commit}, muitas vezes não diretamente relacionadas à mensagem do mesmo.

\section{Estrutura do trabalho}
\label{sec:estruturaTrabalho}
