%%%% CAPÍTULO 1 - INTRODUÇÃO
%%
%% Deve apresentar uma visão global da pesquisa, incluindo: breve histórico, importância e justificativa da escolha do tema,
%% delimitações do assunto, formulação de hipóteses e objetivos da pesquisa e estrutura do trabalho.

%% Título e rótulo de capítulo (rótulos não devem conter caracteres especiais, acentuados ou cedilha)
\chapter{Introdução}\label{cap:introducao}
OBS: A referência do Couloris não tem um "free download", isso é um problema? Além disso, quando devo referenciar a página em que algo se encontra e quando não?

OBS: Coloco a sitação do Tanenbaum em inglês ou português?

OBS: Como citar o título de um bibtex?

OBS: Uso aspas ao colocar o nome de um livro ou coisa do tipo?

No cenário atual da computação, os \gls{sds} desempenham um papel fundamental, alimentando a infraestrutura de serviços e aplicações que impulsionam nosso mundo digital. De acordo com \cite{TanenbaumDistributedSystemsThirdEdition}, um \gls{sd} é definido como "um conjunto de computadores independentes que se apresenta a seus usuários como um sistema único e coerente" (p. 968). Diante disso, emergem desafios inerentes devido à necessidade de apresentar o sistema ao usuário como uma entidade homogênea, embora, em última análise, o sistema seja constituído por diversas partes heterogêneas. A heterogeneidade, como destacada em \textit{Distributed Systems: Concepts and Design} \cite{DistributedSystemsCoulouris}, abrange diversos aspectos, incluindo variações em termos de rede, hardware, sistemas operacionais, linguagens de programação e implementações realizadas por diferentes desenvolvedores.

Essa complexidade inerente aos \gls{sds} os torna um campo desafiador, mas ao mesmo tempo, altamente relevante na computação moderna. Compreender e lidar com essa heterogeneidade e complexidade é essencial para o desenvolvimento, manutenção e escalabilidade de \gls{sd}. Além disso, a crescente dependência desses sistemas em nossa sociedade digital torna o estudo de \gls{sd} fundamental para a formação de profissionais capacitados e para o avanço da qualidade do ensino em Engenharia de Software. É nesse contexto que a investigação sobre a evolução do código em \gls{sd} se destaca, proporcionando uma visão valiosa sobre como esses sistemas complexos evoluem para atender às demandas em constante mudança da era digital.


\section{Objetivos}\label{sec:objetivos}

\subsection{Objetivo geral}\label{subsec:objetivoGeral}
O objetivo geral desta monografia é desenvolver uma heurística para identificar exemplos de código-fonte representativos da evolução de \gls{sds} por meio da análise de métricas de código. Essa heurística será projetada para auxiliar na seleção de exemplos trabalhados que demonstrem como o código em \gls{sds} se adapta e evolui ao longo do tempo. O principal propósito é fornecer materiais de ensino e estudo que contribuam para a formação de estudantes e profissionais em Engenharia de Software, ampliando sua compreensão das complexidades, evolução e desafios dos \gls{sds} em um ambiente cada vez mais distribuído e interconectado.

\subsection{Objetivos específicos}\label{subsec:objetivosEspecificos}
OBS: Colocar no infinitivo. Não é passo a passo, mas subprodutos que, quando completados, cumprem o objetivo geral
\begin{itemize}
    \item Escolher repositório(s) open-source apropriado(s) relacionado(s) à area de \gls{sd}.
    \item Seleção de ferramentas para geração de métricas do(s) repositório(s) selecionado(s).
    \item Seleção das métricas relevantes para a heurística a ser desenvolvida.
    \item Análise dos dados para validação da heurística.
    \item Refinamento da heurística.
    \item Seleção de possíveis bons exemplos e estudo do que é possível retirar de boas refatoraçôes.
\end{itemize}

\section{Justificativa}\label{sec:justificativa}
OBS: Justificativa em cima do seu objetivo.
Explicar que o resultado vai servir para ser usado em sala de aula para contrinbuir para o enriquecimento da universidade.

\section{Estrutura do trabalho}\label{sec:estruturaTrabalho}

OBS: Em algum canto na introdução, discorrer sobre a viabilidade do estudo e dos fatores que podem "inviabilizar" ela como, por exemplo, os dados serem muito poluídos no geral, coisa que é sim