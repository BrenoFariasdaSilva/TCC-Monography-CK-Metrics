%%%% ABSTRACT
%%
%% Versão do resumo para idioma de divulgação internacional.

\begin{abstractutfpr}%% Ambiente abstractutfpr

This research work addressed the evolution of the code in Distributed Systems (DS) through the analysis of code metrics, aiming to create worked examples applicable to teaching Software Engineer (SE). The complexity inherent to DS, crucial in contemporary digital infrastructure, highlights the need to understand the evolution of the code to train qualified professionals. The research was conducted in an exploratory manner, using a qualitative approach to investigate the relationship between code metrics and improvements in DS. The heuristics developed for the selection of representative codes proved to be fundamental in creating worked examples that illustrate, in a practical way, the dynamics of code evolution in DS. Judicious choice of tools such as CK, RefactoringMiner, and PyDriller enabled comprehensive analysis of code metrics and changes over time in relevant repositories such as Apache Kafka and ZooKeeper. The adopted methodology enabled a holistic understanding of the relationship between specific metrics and code improvements, incorporating qualitative elements in the analysis. The results of this aim to contribute to the training of students and professionals in SE, filling a gap in the literature on the creation of worked examples in DS. By providing a deeper understanding of the evolution of the code in DS and offering pedagogically valuable worked examples, we hope to positively impact teaching and professional practice in the field.

\end{abstractutfpr}
