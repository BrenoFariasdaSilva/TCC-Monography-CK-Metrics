%%%% ABSTRACT
%%
%% Versão do resumo para idioma de divulgação internacional.

\begin{abstractutfpr}%% Ambiente abstractutfpr

This monograph presents a research proposal for \gls{tcc2}, addressing the evolution of code in \gls{sds} through code metrics analysis. The main objective is to develop worked examples applicable to \gls{es} education. The intrinsic complexity of \gls{sds}, playing a crucial role in contemporary computational infrastructure, emphasizes the importance of understanding code evolution to educate qualified professionals. The theoretical framework is based on understanding the complexity of \gls{sds} in the digital era, justifying the need to study code evolution to train qualified professionals in \gls{es}. The relevance of the selected code metrics is highlighted, providing a comprehensive view of code quality and addressing different aspects. It also explores the concept of what a worked example is and its applications in \gls{es} and \gls{sds}. The research adopts an exploratory and qualitative approach to investigate the relationship between code metrics and improvements in \gls{sds}. The heuristic to be developed in \gls{tcc2} for the selection of representative codes is crucial for creating worked examples that illustrate the dynamics of code evolution in \gls{sds} practically. The careful choice of refactoring and code analysis tools will allow a comprehensive analysis of code metrics and changes over time in repositories of relevant \gls{sds} source code. The adopted methodology enables a holistic understanding of the relationship between specific metrics and code improvements, incorporating qualitative elements into the analysis. The results of this research aim to contribute to the education of students and professionals in \gls{es}, filling a gap in the literature, as there are no directly comparable projects to serve as a reference for creating worked examples for \gls{es} using \gls{sds} as the object of study. Despite not-so-promising preliminary results, representing expected challenges, these challenges provide opportunities for ongoing research. Moreover, the schedule outlined for \gls{tcc2} incorporates elements already completed at this stage of the project. In this early stage of research, represented only by a proposal, the intrinsic challenges and limitations of the project are clearly outlined, with the expectation of positively impacting \gls{es} education and professional practice.

\end{abstractutfpr}
