%%%% AGRADECIMENTOS
%%
%% Texto em que o autor faz agradecimentos dirigidos àqueles que contribuíram de maneira relevante à elaboração do trabalho.

\begin{agradecimentos}%% Ambiente agradecimentos

Certamente não haveria forma diferente de começar os meus agradecimentos sem expressar minha profunda gratidão ao meu orientador, Prof. Dr. Marco Aurélio Graciotto Silva, e ao co-orientador, Prof. Dr. Rodrigo Campiolo, pelas imensuráveis orientações e feedback constante ao longo deste projeto de pesquisa. A dedicação e apoio de ambos foram fundamentais para o sucesso deste trabalho e por estarem presentes em minha trajetória durante a graduação.

Além disso, quero agradecer ao Prof. Dr. Luiz Arthur, que não esteve diretamente envolvido no projeto, mas que desempenhou um papel crucial como professor, mas também como amigo. Sua amizade e apoio constante tornaram minha jornada de aprendizado uma experiência enriquecedora. Suas conversas e mentorias moldaram minha perspectiva e contribuíram extraordinariamente para o meu crescimento como estudante e como pessoa.

Ao Prof. Dr. Igor Wiese, agradeço seu apoio e sua disposição para compartilhar conhecimento e incentivar minha participação em atividades extracurriculares, abrindo portas que foram fundamentais para o meu desenvolvimento como aluno nessa instituição.

Gostaria também de agradecer a minha namorada, Amanda Carvalho, visto que, todos os dias, ela esteve ao meu lado, oferecendo apoio inabalável, especialmente nos momentos em que duvidava das minhas capacidades, deixando-se sempre disponível para dividir as minhas dores e anseios com ela.

Gostaria de expressar minha profunda gratidão à minha mãe, Márcia Farias, pois, mesmo mantendo pouco contato, sempre esteve preocupada em saber se estou bem e feliz. Agradeço por ser uma fonte inesgotável de amor, por sempre me incentivar a alcançar os meus objetivos que, independente do quão grande possam chegar a ser, sempre me apoiou e nunca duvidou de mim.

Gostaria de enviar um abraço especial aos meus amigos incríveis de Portugal, Bernardo Louro, Diogo Lopes e Simão Farias. Mesmo estando a mais de 7 mil km de distância, vocês fizeram parte de cada passo dessa jornada maluca. Vocês são mais do que amigos; são a prova de que as verdadeiras conexões resistem a qualquer distância. Obrigado por serem parte da minha vida de uma maneira tão única e significativa!

Por último, mas definitivamente não menos importante, dedico um agradecimento especial ao meu pai, Manoel Campos, minha maior inspiração de valor incomensurável. Seu apoio constante e palavras de incentivo não apenas foram um fator motivador crucial, mas também por sempre demonstrar orgulho e apontar que estou no caminho certo. Agradeço sinceramente por ter um pai cuja presença é insubstituível, tornando minha jornada ainda mais significativa.

A todos os que, de alguma forma, contribuíram para a realização desta pesquisa, meu sincero agradecimento.

Se o aluno recebeu bolsa de fomento à pesquisa, informar o nome completo da agência de fomento. Ex: Capes, CNPq, Fundação Araucária, UTFPR, etc. Incluir o número do projeto após a agência de fomento. Este item deve ser o último.

\end{agradecimentos}
