%%%% RESUMO
%%
%% Apresentação concisa dos pontos relevantes de um texto, fornecendo uma visão rápida e clara do conteúdo e das conclusões do
%% trabalho.

\begin{resumoutfpr}%% Ambiente resumoutfpr

Este trabalho de pesquisa abordou a evolução do código em \gls{sds} por meio da análise de métricas de código, visando criar exemplos trabalhados aplicáveis ao ensino de \gls{es}. A complexidade inerente aos \gls{sds}, cruciais na infraestrutura digital contemporânea, destaca a necessidade de compreender a evolução do código para formar profissionais qualificados. A pesquisa foi conduzida de maneira exploratória, utilizando uma abordagem qualitativa para investigar a relação entre métricas de código e melhorias em \gls{sds}. A heurística desenvolvida para a seleção de códigos representativos mostrou-se fundamental na criação de exemplos trabalhados que ilustram, de forma prática, a dinâmica da evolução do código em \gls{sds}. A escolha criteriosa de ferramentas, como CK, RefactoringMiner e PyDriller, permitiu uma análise abrangente das métricas de código e das mudanças ao longo do tempo em repositórios relevantes, como Apache Kafka e ZooKeeper. A metodologia adotada possibilitou uma compreensão holística da relação entre métricas específicas e melhorias no código, incorporando elementos qualitativos na análise. Os resultados desta pretendem contribuir para a formação de estudantes e profissionais em \gls{es}, preenchendo uma lacuna na literatura sobre a criação de exemplos trabalhados em \gls{sds}. Ao proporcionar uma compreensão mais profunda da evolução do código em \gls{sds} e oferecer exemplos trabalhados pedagogicamente valiosos, espera-se impactar positivamente o ensino e a prática profissional na área.

\end{resumoutfpr}
