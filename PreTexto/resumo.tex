%%%% RESUMO
%%
%% Apresentação concisa dos pontos relevantes de um texto, fornecendo uma visão rápida e clara do conteúdo e das conclusões do
%% trabalho.

\begin{resumoutfpr}%% Ambiente resumoutfpr

Esta monografia apresenta uma proposta de pesquisa para o \gls{tcc2}, abordando a evolução do código em \gls{sds} por meio da análise de métricas de código. O objetivo principal é desenvolver exemplos trabalhados que sejam aplicáveis ao ensino de \gls{es}. A complexidade intrínseca dos \gls{sds}, que desempenham papel crucial na infraestrutura computacional contemporânea, ressalta a importância de compreender a evolução do código para formar profissionais qualificados. O referencial teórico deste trabalho fundamenta-se na compreensão da complexidade dos \gls{sds} na era digital, justificando a necessidade de estudar a evolução do código para formar profissionais qualificados em \gls{es}. Destaca-se a relevância da seleção de métricas de código já selecionadas, as quais fornecem uma visão abrangente da qualidade do código, abordando diferentes aspectos, além de explorar o conceito do que é um exemplo trabalhado e suas aplicações na \gls{es} e em \gls{sds}. A pesquisa adotou uma abordagem exploratória e qualitativa para investigar a relação entre métricas de código e melhorias em \gls{sds}. A heurística a ser desenvolvida no \gls{tcc2} para a seleção de códigos representativos é fundamental na criação de exemplos trabalhados que ilustrem, de maneira prática, a dinâmica da evolução do código em \gls{sds}. A escolha criteriosa de ferramentas de refatoração e análise de código permitirá uma análise abrangente das métricas de código e das mudanças ao longo do tempo em repositórios de código-fonte de \gls{sds} relevantes. A metodologia adotada possibilita uma compreensão holística da relação entre métricas específicas e melhorias no código, incorporando elementos qualitativos na análise. Os resultados desta pesquisa buscam contribuir para a formação de estudantes e profissionais em \gls{es}, preenchendo uma lacuna na literatura, pois não existem projetos diretamente comparáveis para servir de referência, sobre a criação de exemplos trabalhados para a \gls{es}, utilizando \gls{sds} como objeto de estudo. Apesar de resultados prévios não tão animadores, representando desafios esperados, esses desafios oferecem oportunidades para a continuidade da pesquisa. Não suficiente, o cronograma exposto para o \gls{tcc2} incorpora elementos já realizados ainda nesta etapa do projeto. Neste estágio inicial da pesquisa, representado apenas por uma proposta, os desafios e limitações intrínsecos ao projeto estão claramente delineados, além de que espera-se impactar positivamente o ensino e a prática profissional na área de \gls{es}.

\end{resumoutfpr}
